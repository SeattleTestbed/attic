\section{Introduction}

Cloud computing is rapidly increasing in popularity as many organizations
outsource hardware and maintenance and instead focus on 
software~\cite{AmazonEC2, AmazonS3, GoogleAppEngine, GoogleDocs, HPCloud, 
IBMBlueCloud, MicrosoftMemo}.   
%Amazon
%has built an infrastructure for Elastic Cloud Computing (EC2)~\cite{AmazonEC2} 
%on top of RedHat servers.   Google has provided a programming API and platform
%called AppEngine~\cite{GoogleAppEngine} that transparently provides scalability
%while hiding locality information from applications.   Google also provides
%applications like Google Docs~\cite{GoogleDocs} which are built using cloud
%computing principles.  Other major industry players like 
%Microsoft~\cite{MicrosoftMemo} and IBM~\cite{IBMBlueCloud} have announced their
%own cloud computing platforms and HP, Intel, and Yahoo!~\cite{HPCloud} have 
%announced a joint pushes for cloud computing research.   
%%%%%
%%%% An attempt where I talk about client software as well as cloud differences
%However, despite all of the focus on cloud computing, there
%is a lot of disparity in industry in terms of what cloud computing 
%really means.   RedHat / Amazon's EC2~\cite{AmazonEC2} provides cloud
%computing as a collection of Linux boxes along with
%storage functionality~\cite{AmazonS3} while leaving the client interaction
%with the cloud up to the developer.   Google provides automatic scaling of
%cloud computing programs written to a custom programming 
%API~\cite{GoogleAppEngine} and tends to have clients access cloud resources 
%using their web browser~\cite{GoogleDocs}.   Microsoft views it as a 
%virtualization layer between the hardware and OS and is releasing 
%developer toolkits to be used in a ``software plus service'' architecture where
%the client runs custom cloud software on their local 
%machine~\cite{MicrosoftMemo}.
%%%%%
However, despite the attention, there is a lot of disparity in what
cloud computing means. RedHat / Amazon's EC2~\cite{AmazonEC2} provides
cloud computing as a collection of Linux machines with storage
functionality~\cite{AmazonS3}. Google's platform for cloud computing
hides locality and scalability issues from the programmer who writes
programs to a custom programming API~\cite{GoogleAppEngine}. Microsoft
views it as a virtualization layer between the hardware and the OS and
is releasing a developer toolkit for providing the user with
``software plus service.''~\cite{MicrosoftMemo}

We provide an educational platform called Seattle that is a common
denominator for a wide range of these definitions. Seattle's simple to
learn programming language, a subset of the Python language, is
expressive enough to allow students to build algorithms for
inter-machine interaction (like a global store or a DHT). As a result,
Seattle is useful in many pedagogical contexts ranging from courses in
cloud computing, networking, and distributed systems, to parallel
programming, grid computing, and peer-to-peer computing.

Seattle is a community-driven effort that depends on resources donated
by users of the software (and as such is free to use). A user can
install Seattle onto their personal computer to enable Seattle
programs to run using a portion of the computer's resources. Seattle
programs are sandboxed and securely isolated from other programs
running on the same computer. Seattle provides hard resource
guarantees that an erroneous or malicious program cannot circumvent.

In addition, Seattle runs on a variety of different operating systems
and architectures including Windows, Mac OS-X, Linux, FreeBSD, and even
portable devices like Nokia N800s and jail broken iPhones. Code written
for Seattle is automatically (and transparently) portable to different
architectures and runs the same across all systems.

Seattle has a preexisting base of installed computers and is already
widely deployed on almost one thousand computers that are spread
across hundreds of universities worldwide. Seattle users can run their
programs on computers spanning the Internet -- a feature that is
currently being used by several classes at major universities.

%While Seattle is useful as a cloud computing platform, there are
%other important pedagogical concepts that Seattle can illustrate.
%The first concept is that the characteristics of links between computing 
%resources 
%matters.  In other words, two computers in a data center behave differently 
%than two computers connected by a trans-atlantic cable.   The second concept is
%that the grouping of resources is important.   It is possible to efficiently 
%solve different sets of problems with 100\% of one CPU than 10\% of 10 
%computers.   The third concept is that failure handling is important.   In 
%globally distributed applications, communications failures due to routing
%black holes or DNS issues are commonplace~\cite{Katz_NSDI_2008}.   Applications
%that seek to provide high uptime must handle failures well.

%\subsection{Map}
This paper describes the architecture of the Seattle cloud computing
platform (Section~\ref{sec-architecture}) including the
programming API (Section~\ref{sec-API}), 
the sandboxing mechanism (Section~\ref{sec-repy}), 
the control of sandboxes on a
host computer (Section~\ref{sec-nodemanager}), and the way in which
students manage their running programs
(Section~\ref{sec-experimentmanager}). Following this, we describe the
computational resources available to classes using Seattle and how we
expect this platform to grow in the future
(Section~\ref{sec-deployment}).  Next, we provide some example
assignments to demonstrate how Seattle can be used in courses
(Section~\ref{sec-assignments}). We then discuss related work
(Section~\ref{sec-relatedwork}) and conclude
(Section~\ref{sec-conclusion}).



