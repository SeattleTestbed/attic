\section{Seattle in the Classroom}
\label{sec-assignments}

This section describes educational resources available to the student and 
instructor.   The resources are divided into two areas.   There is a student
portal which contains resources for students who are learning the basics of
Seattle.   There is also an educator portal that contains resources for 
educators to help them use Seattle in the classroom.   

\subsection{Student Resources}
To aid students learning Seattle, we provide a student portal with
tutorials describing how to program and use Seattle, documentation for the API,
a resources and restrictions guide, and other documentation.   The tutorials 
provide code snippets demonstrating the API, and explains how to use the 
experiment manager to perform different tasks.
Our experience has shown that students can quickly learn the Python
programming language~\cite{611980, Cappos_WORLDS_2005} and quickly learn
to program in Seattle.   Our experience is that undergraduates who had no 
previous networking experience can implement programs like overlay 
multicast and TCP forwarding with a few hours of effort after completing 
the tutorial.
% JAC: Should this be removed? Is mentioning ease of learning a distraction?
% ib: i think this should stay, an important characteristic for teaching with Seattle

To illustrate how easy it is to program Seattle, 
A popular first project for networking students is the Echo
client/server. The two \emph{complete} Seattle programs are both
concise and simple. The echo client is just 6 lines of code:

{\scriptsize
\begin{verbatim}
# Handle an incoming message
def got_reply(srcip, srcport, mess, ch):
  print 'received:',mess,"from",srcip,srcport

if callfunc == 'initialize':
  # when a message arrives on my IP, port 43210, 
  # start an event to call the function 'got_reply'
  recvmess(getmyip(), 43210, got_reply)  
  # send a hello message to my IP, port 54321
  sendmess(getmyip(), 54321, 'hello', getmyip(), 43210)  
  # exit in one second
  settimer(1,exitall,())
\end{verbatim}
}

The echo server consists of just 4 lines of code:

{\scriptsize
\begin{verbatim}
# Handle an incoming message
def got_message(srcip, srcport, mess, ch):
  sendmess(srcip,srcport,mess)

if callfunc == 'initialize':
  recvmess(getmyip(), 54321, got_message)  
\end{verbatim}
}


Furthermore, a first project for many students in distributed systems is to 
measure the connectivity characteristics between pairs of
computers~\cite{AllPairsPing}.  The Appendix lists a complete 30 line Seattle
program that performs an all-pairs-ping and displays its results in a webpage 
when contacted.


\subsection{Educator Resources}

To aid instructors in integrating Seattle into their existing curriculum, 
we also provide an educator portal.   The first item on the educator 
portal is a description of instructor experiences with Seattle.   After
Seattle has been used, we have asked the students to fill out a survey 
outlining how they felt the platform impacted their experience.   We recorded
the results on the web page.

Besides information about experiences with Seattle, the educator portal also 
contains course materials such as handouts and example assignments.   
One special-purpose assignment called the TakeHomeAssignment requires 
no programming and takes about 1 hour to do.   The purpose of the 
TakeHomeAssignment is to show the student or instructor the practical effects 
of non-transitive connectivity and NATs on the Internet today, while 
introducing them to Seattle.

In addition, the educator portal provides a collection of ready-to-go 
programming assignments for use with Seattle.   The assignments include implementing a 
reliable protocol on top of UDP, performing overlay routing using link-state 
routing on the Internet, building application level services like a webserver, and 
understand layering of services by creating a chat server that operates 
over HTTP.   

The educator portal also provides assignment ideas that are appropriate 
for course long projects or graduate assignments.   For example, students 
may implement a DHT (such as Chord~\cite{Stoica_SIGCOMM_2001}) to better 
understand non-transitive connectivity.   A simple implementation will work
well on LAN environments but will fail horribly on the Internet due to 
non-transitive connectivity and other network effects.   After measuring
these effects and then understanding the reason behind Chord's poor 
performance, the students can discuss solutions to these
problems. Students can then implement and test these solutions to achieve
better performance and reliability. 
This assignment emphasizes good software engineering practices (since code is
reused), that test and deployment environments may differ significantly, 
and encourages students to come up with unique solutions to the problem yet 
is easily evaluated using a small set of metrics.


