\section{Deployment}
\label{sec-deployment}

Since it is unclear what the future of cloud computing will be, we are
interested in providing students with the most diverse set of
resources possible. Seattle runs on Linux, FreeBSD, Mac OS-X, XO (one
laptop per child)\footnote[1]{Some threading libraries and operating
  systems do not provide accurate CPU and/or memory information. As a
  result, certain resources cannot be effectively sandboxed}, and
Windows platforms. Seattle also runs on mobile devices like the
iPhone\footnotemark[1] (if jail broken) and Nokia N800\footnotemark[1].
We are interested in adding support for other platforms as users
express interest in running Seattle on new platforms.

In addition to platform diversity, network connection diversity is
also important. Seattle is already widely deployed at universities
around the world. An instance of Seattle is running on each 
PlanetLab~\cite{PlanetLab} node, giving a presence on close to one thousand
nodes at hundreds of universities.  

PlanetLab provides access to computers at a large number of well
connected locations, although most only have two computers per site
available. As we are now publicly releasing Seattle for educational
use, we expect to see an increase in resource diversity.  We
anticipate that universities will install Seattle on most of the
computers in their lab. Such deployments will provide a distributed
environment that effectively emulates cluster computing. We also
anticipate that many students will load Seattle on their home
computers. This will allow for emulation of peer-to-peer computing as
home computers will typically have connectivity characteristics
representative of the average Internet user.

We believe that the diversity of platforms and network connections
enables a wide range of pedagogical uses. Students can experiment with
cluster computing, grid computing, peer-to-peer, and cloud computing
by simply varying where their program is deployed. Additionally, the
same program will run in any of these environments. Of course, the
characteristics of the environment will determine the efficiency and
scalability of the employed distributed algorithm --- a crucial
distributed systems lesson for students.

