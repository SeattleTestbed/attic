\abstract

%As many organizations outsource hardware and maintainence and instead
%focus on software, % IB: do we need give a reason for the transition?

Cloud computing is rapidly increasing in popularity. Companies such as
RedHat, Microsoft, Amazon, Google, and IBM are increasingly funding
cloud computing infrastructure and research, making it important for
students to gain the necessary skills to work with cloud-based
resources. This paper presents a free, educational research platform
called Seattle that is community-driven, a common denominator for
diverse platform types, and is broadly deployed.

Seattle is community-driven --- universities donate available compute
resources on multi-user machines to the platform. These donations can
come from systems with a wide variety of operating systems and
architectures, removing the need for a dedicated infrastructure.
% to support Seattle.

Seattle is also surprisingly flexible and supports a variety of
pedagogical uses because as a platform it represents a common denominator 
for cloud computing, grid computing, peer-to-peer networking,
distributed systems, and networking. Seattle programs are portable.
Students' code can run across different operating systems and
architectures without change, while the Seattle programming language
is expressive enough for experimentation at a fine-grained level. Our
current deployment of Seattle consists of about one thousand computers
that are distributed around the world. We invite the computer science
education community to employ Seattle in their courses.

% mention planetlab

% arvind's comments:
%Community-driven (resources are provided by the community).
%
%Common denominator for wide range of computing types
%
%Free
%
%Broad deployment (many hundreds of systems)

%Cloud computing is rapidly increasing in popularity as many organizations
%outsource hardware and maintainence and instead focus on software.   With
%major industry players like RedHat, Microsoft, Amazon, Google, and IBM
%investing heavily in cloud computing infrastructure and research, it is 
%important that students gain the skills they need to work with cloud-based 
%resources.   
%We have built a cloud computing platform called Seattle for the
%purpose of allowing students to experiment with cloud computing in a safe,
%portable, and cost-free environment.   Computers that run Seattle are protected
%from buggy and malicious programs by a sandboxing mechanism that provides 
%security and performance isolation.  Students that use Seattle can write 
%programs that use resources on around one thousand computers running at 
%hundreds of universities distributed around the world.   An instructor can
%use Seattle to demonstrate the difference between cluster computing 
%(possibly remote computers with low inter-computer latency), cloud 
%computing (collaborative computers where at least some systems have low
%latency to the source), peer-to-peer computing (randomly distributed 
%computers), and other in-between scenarios.
