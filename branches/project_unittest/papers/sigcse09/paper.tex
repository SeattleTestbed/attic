% THIS IS SIGPROC-SP.TEX - VERSION 2.9
% WORKS WITH V3.0SP OF ACM_PROC_ARTICLE-SP.CLS
% MARCH 2007
%
% It is an example file showing how to use the 'acm_proc_article-sp.cls' V3.0SP
% LaTeX2e document class file for Conference Proceedings submissions.
% ----------------------------------------------------------------------------------------------------------------
% This .tex file (and associated .cls V3.0SP) *DOES NOT* produce:
%       1) The Permission Statement
%       2) The Conference (location) Info information
%       3) The Copyright Line with ACM data
%       4) Page numbering
% ---------------------------------------------------------------------------------------------------------------
% It is an example which *does* use the .bib file (from which the .bbl file
% is produced).
% REMEMBER HOWEVER: After having produced the .bbl file,
% and prior to final submission,
% you need to 'insert'  your .bbl file into your source .tex file so as to provide
% ONE 'self-contained' source file.
%
% Questions regarding SIGS should be sent to
% Adrienne Griscti ---> griscti@acm.org
%
% Questions/suggestions regarding the guidelines, .tex and .cls files, etc. to
% Gerald Murray ---> murray@acm.org
%
% For tracking purposes - this is V2.9SP - MARCH 2007

\documentclass{sig-alternate}
\usepackage{verbatim}
\usepackage{url}

%\pagenumbering{arabic}

%-- Begin patch area for accents in 'Author Block' area - may be needed by some authors / but not all
\DeclareFixedFont{\auacc}{OT1}{phv}{m}{n}{12}   % Needed for "Author Block" accents - Patch by Gerry 3/21/07
\DeclareFixedFont{\afacc}{OT1}{phv}{m}{n}{10}   % Needed for "Author Block" accents in the affiliation/address line - Patch by Gerry 3/21/07
%--
\newtheorem{remark}{Remark}

% I'm using this so I can easily mark parts that are moved to the TR...
\newcommand{\movetotr}[1]{{}}

\begin{document}
\conferenceinfo{SIGCSE'09,} {March 3--7, 2009, Chattanooga, Tennessee, USA.} 

\CopyrightYear{2009}

\crdata{978-1-60558-183-5/09/03} 

\title{Seattle: A Platform for Educational Cloud Computing\thanks{This work 
was partially supported by NSF Grant CNS-0834243}
}
%
% You need the command \numberofauthors to handle the 'placement
% and alignment' of the authors beneath the title.
%
% For aesthetic reasons, we recommend 'three authors at a time'
% i.e. three 'name/affiliation blocks' be placed beneath the title.
%
% NOTE: You are NOT restricted in how many 'rows' of
% "name/affiliations" may appear. We just ask that you restrict
% the number of 'columns' to three.
%
% Because of the available 'opening page real-estate'
% we ask you to refrain from putting more than six authors
% (two rows with three columns) beneath the article title.
% More than six makes the first-page appear very cluttered indeed.
%
% Use the \alignauthor commands to handle the names
% and affiliations for an 'aesthetic maximum' of six authors.
% Add names, affiliations, addresses for
% the seventh etc. author(s) as the argument for the
% \additionalauthors command.
% These 'additional authors' will be output/set for you
% without further effort on your part as the last section in
% the body of your article BEFORE References or any Appendices.

%\numberofauthors{5} %  in this sample file, there are a *total*

% of EIGHT authors. SIX appear on the 'first-page' (for formatting
% reasons) and the remaining two appear in the \additionalauthors section.
%
\author{
%Author Names Removed for Anonymous Submission
Justin Cappos~~~~Ivan Beschastnikh~~~~Arvind Krishnamurthy~~~~Tom Anderson\\[1ex]
       \affaddr{Department of Computer Science and Engineering, University of Washington}\\
       \affaddr{Seattle, WA 98105, U.S.A.}\\
       \email{\{justinc, ivan, arvind, tom\}@cs.washington.edu}
}
\date{}
%\hyphenation{Com-put-er-Comm-uni-ca-tion Comm-uni-ca-tion}

\maketitle
\abstract

%As many organizations outsource hardware and maintainence and instead
%focus on software, % IB: do we need give a reason for the transition?

Cloud computing is rapidly increasing in popularity. Companies such as
RedHat, Microsoft, Amazon, Google, and IBM are increasingly funding
cloud computing infrastructure and research, making it important for
students to gain the necessary skills to work with cloud-based
resources. This paper presents a free, educational research platform
called Seattle that is community-driven, a common denominator for
diverse platform types, and is broadly deployed.

Seattle is community-driven --- universities donate available compute
resources on multi-user machines to the platform. These donations can
come from systems with a wide variety of operating systems and
architectures, removing the need for a dedicated infrastructure.
% to support Seattle.

Seattle is also surprisingly flexible and supports a variety of
pedagogical uses because as a platform it represents a common denominator 
for cloud computing, grid computing, peer-to-peer networking,
distributed systems, and networking. Seattle programs are portable.
Students' code can run across different operating systems and
architectures without change, while the Seattle programming language
is expressive enough for experimentation at a fine-grained level. Our
current deployment of Seattle consists of about one thousand computers
that are distributed around the world. We invite the computer science
education community to employ Seattle in their courses.

% mention planetlab

% arvind's comments:
%Community-driven (resources are provided by the community).
%
%Common denominator for wide range of computing types
%
%Free
%
%Broad deployment (many hundreds of systems)

%Cloud computing is rapidly increasing in popularity as many organizations
%outsource hardware and maintainence and instead focus on software.   With
%major industry players like RedHat, Microsoft, Amazon, Google, and IBM
%investing heavily in cloud computing infrastructure and research, it is 
%important that students gain the skills they need to work with cloud-based 
%resources.   
%We have built a cloud computing platform called Seattle for the
%purpose of allowing students to experiment with cloud computing in a safe,
%portable, and cost-free environment.   Computers that run Seattle are protected
%from buggy and malicious programs by a sandboxing mechanism that provides 
%security and performance isolation.  Students that use Seattle can write 
%programs that use resources on around one thousand computers running at 
%hundreds of universities distributed around the world.   An instructor can
%use Seattle to demonstrate the difference between cluster computing 
%(possibly remote computers with low inter-computer latency), cloud 
%computing (collaborative computers where at least some systems have low
%latency to the source), peer-to-peer computing (randomly distributed 
%computers), and other in-between scenarios.

% A category with the (minimum) three required fields
\category{K.3.2}{Computer and Information Science Education}{Computer science education}
\category{C.2.4}{Computer-Communi-cation Networks}{Distributed Systems}[client/server, distributed applications]
\category{C.4}{Computer Systems Organization}{Performance of Systems}[design studies, measurement techniques]

%A category including the fourth, optional field follows...\category{D.2.8}{Software Engineering}{Metrics}[complexity measures, performance measures]
\terms{Experimentation, Measurement, Performance, Security}

\keywords{Cloud computing, peer-to-peer computing, cluster computing, distributed computing}
\section{Introduction}

Cloud computing is rapidly increasing in popularity as many organizations
outsource hardware and maintenance and instead focus on 
software~\cite{AmazonEC2, AmazonS3, GoogleAppEngine, GoogleDocs, HPCloud, 
IBMBlueCloud, MicrosoftMemo}.   
%Amazon
%has built an infrastructure for Elastic Cloud Computing (EC2)~\cite{AmazonEC2} 
%on top of RedHat servers.   Google has provided a programming API and platform
%called AppEngine~\cite{GoogleAppEngine} that transparently provides scalability
%while hiding locality information from applications.   Google also provides
%applications like Google Docs~\cite{GoogleDocs} which are built using cloud
%computing principles.  Other major industry players like 
%Microsoft~\cite{MicrosoftMemo} and IBM~\cite{IBMBlueCloud} have announced their
%own cloud computing platforms and HP, Intel, and Yahoo!~\cite{HPCloud} have 
%announced a joint pushes for cloud computing research.   
%%%%%
%%%% An attempt where I talk about client software as well as cloud differences
%However, despite all of the focus on cloud computing, there
%is a lot of disparity in industry in terms of what cloud computing 
%really means.   RedHat / Amazon's EC2~\cite{AmazonEC2} provides cloud
%computing as a collection of Linux boxes along with
%storage functionality~\cite{AmazonS3} while leaving the client interaction
%with the cloud up to the developer.   Google provides automatic scaling of
%cloud computing programs written to a custom programming 
%API~\cite{GoogleAppEngine} and tends to have clients access cloud resources 
%using their web browser~\cite{GoogleDocs}.   Microsoft views it as a 
%virtualization layer between the hardware and OS and is releasing 
%developer toolkits to be used in a ``software plus service'' architecture where
%the client runs custom cloud software on their local 
%machine~\cite{MicrosoftMemo}.
%%%%%
However, despite the attention, there is a lot of disparity in what
cloud computing means. RedHat / Amazon's EC2~\cite{AmazonEC2} provides
cloud computing as a collection of Linux machines with storage
functionality~\cite{AmazonS3}. Google's platform for cloud computing
hides locality and scalability issues from the programmer who writes
programs to a custom programming API~\cite{GoogleAppEngine}. Microsoft
views it as a virtualization layer between the hardware and the OS and
is releasing a developer toolkit for providing the user with
``software plus service.''~\cite{MicrosoftMemo}

We provide an educational platform called Seattle that is a common
denominator for a wide range of these definitions. Seattle's simple to
learn programming language, a subset of the Python language, is
expressive enough to allow students to build algorithms for
inter-machine interaction (like a global store or a DHT). As a result,
Seattle is useful in many pedagogical contexts ranging from courses in
cloud computing, networking, and distributed systems, to parallel
programming, grid computing, and peer-to-peer computing.

Seattle is a community-driven effort that depends on resources donated
by users of the software (and as such is free to use). A user can
install Seattle onto their personal computer to enable Seattle
programs to run using a portion of the computer's resources. Seattle
programs are sandboxed and securely isolated from other programs
running on the same computer. Seattle provides hard resource
guarantees that an erroneous or malicious program cannot circumvent.

In addition, Seattle runs on a variety of different operating systems
and architectures including Windows, Mac OS-X, Linux, FreeBSD, and even
portable devices like Nokia N800s and jail broken iPhones. Code written
for Seattle is automatically (and transparently) portable to different
architectures and runs the same across all systems.

Seattle has a preexisting base of installed computers and is already
widely deployed on almost one thousand computers that are spread
across hundreds of universities worldwide. Seattle users can run their
programs on computers spanning the Internet -- a feature that is
currently being used by several classes at major universities.

%While Seattle is useful as a cloud computing platform, there are
%other important pedagogical concepts that Seattle can illustrate.
%The first concept is that the characteristics of links between computing 
%resources 
%matters.  In other words, two computers in a data center behave differently 
%than two computers connected by a trans-atlantic cable.   The second concept is
%that the grouping of resources is important.   It is possible to efficiently 
%solve different sets of problems with 100\% of one CPU than 10\% of 10 
%computers.   The third concept is that failure handling is important.   In 
%globally distributed applications, communications failures due to routing
%black holes or DNS issues are commonplace~\cite{Katz_NSDI_2008}.   Applications
%that seek to provide high uptime must handle failures well.

%\subsection{Map}
This paper describes the architecture of the Seattle cloud computing
platform (Section~\ref{sec-architecture}) including the
programming API (Section~\ref{sec-API}), 
the sandboxing mechanism (Section~\ref{sec-repy}), 
the control of sandboxes on a
host computer (Section~\ref{sec-nodemanager}), and the way in which
students manage their running programs
(Section~\ref{sec-experimentmanager}). Following this, we describe the
computational resources available to classes using Seattle and how we
expect this platform to grow in the future
(Section~\ref{sec-deployment}).  Next, we provide some example
assignments to demonstrate how Seattle can be used in courses
(Section~\ref{sec-assignments}). We then discuss related work
(Section~\ref{sec-relatedwork}) and conclude
(Section~\ref{sec-conclusion}).




\section{Architecture}
\label{sec-architecture}

To use Seattle, the instructor creates an account on our website and obtains
an installer.   The machines that run the installer (such as computers in 
the universities computer lab) donate resources that are credited to the 
instructor.   The instructor can then obtain resources on machines around the 
world.   As of December 1st, 2008 the current policy is that for each donation,
the instructor receives resources on ten other computers.   However, the 
instructor can delegate those resources either directly to students or
to TAs who do more fine-grained delegation.   Students and TAs download
a toolkit and then experiment with their resources.

Seattle's architecture is comprised of several components. At the
lowest level the \emph{sandbox} component guarantees security and
resource control for an individual program. Programs are written to
the Seattle API in a subset of the Python programming language. This
API provides portable access to low level operations (like opening
files and sending messages). At a higher layer, the \emph{node
  manager} determines which sandboxed programs get to run on the local
computer. A public key infrastructure is used to authorize control
over sandboxed programs. Lastly, the \emph{experiment manager} lets
students control their program instances across computers.

\subsection{Seattle API}
\label{sec-API}

Seattle provides a programming API for low-level operations (like
writing to files or sending network messages) and maintains program
portability using an abstraction layer. Platform specific code below
this abstraction layer handles non-portable operations enabling
unmodified programs to run on a wide variety of
platforms. % This also allows the majority of the sandbox to be reused
%across different architectures.


The API consists of five categories: file, network, timer, locking,
and miscellaneous. The file API calls enable limited access to the
local computer's persistent storage (interacting only with files in a 
single directory). 
%the calls to
%create, delete, open, read and write sandboxed files, as well as calls
%to list directory contents in the sandbox. 
The network API calls
provide the local IP address, perform a DNS lookup, enable sending and
receiving of UDP messages, and managing of and communicating over TCP
connections. The timer API calls enable the programmer to put the
current thread to sleep and to schedule functions to be called at
later times. For example, the programmer can register an event to
periodically send a heartbeat message to another computer. The locking
functions allow the programmer to handle concurrency in their program
(as common state may be accessed and modified by multiple threads at
the same time). The miscellaneous API calls allow the programmer to
exit the program, to generate random numbers, and to provide the
amount of time the program has been running.

% initiation/termination of and listening for TCP connections

\subsection{The Sandbox}
\label{sec-repy}

The sandbox's primary goal is to securely execute user code. There are
two aspects to this --- preventing insecure actions and limiting
resource consumption. To prevent insecure actions the sandbox hooks into 
the Python parser and reads the program's parse tree. Only actions that
the sandbox can verify as safe may execute.

To control resource consumption on the host the sandbox interposes on
all API calls made by a program. The sandbox monitors the overall use
of resources like CPU, memory, and disk space to ensure the program
does not exceed its bounds. Each API call that uses a monitored
resource is evaluated before being granted or denied the resource. The
sandbox also restricts the rate at which API calls are performed.

The restrictions and resource limits of the sandbox are configurable and
may restrict different programs in different ways. For example, one 
program may only be
allowed to receive UDP packets on port 11111, while another program
may be restricted to receiving UDP packets on port 22222. This enables
multiple sandboxes on the same computer to host programs controlled by
different users.

%We provide an interface where the program can execute unsafe operations (such 
%as open files, send network traffic, etc.).   Our sandbox verifies that the 
%program is performing a safe operation (like opening a file in the program's 
%sandbox) instead of a malicious action (like opening the user's credit card 
%information).   This is done by fine grained checks on the arguments passed 
%to individual calls.   Once again we err on the side of caution (for example, 
%restricting the names of files the program can open) to prevent the program 
%from escaping the sandbox.
%
%However, not every safe action should be allowed by every program.   It may 
%be that two programs would like to run on the same system and each program 
%has been allocated its own port.   The programs should not be able to use 
%the other program's port or other ports on the system.   To prevent this 
%repy has per-program fine-grained restrictions that control the use of 
%calls.   Program A can be restricted to port 12345 while Program B is 
%restricted to port 54321.   Similarly, a Program C that does not need 
%to write files to the disk can be prevented from using file operations all 
%together!   This meets our design goal of a multi-user programming environment
%since students are isolated from each other.
%
%
%\subsubsection{Resource Limits}
%\label{sec-resourcelimits}
%
%Another important type of isolation is resource isolation.   One program that 
%runs on a computer should not be able to consume enough resources to 
%impact the execution of other programs or the host computer.   To this end, 
%repy limits resource consumption through several mechanisms.   For 
%resources that renew themselves over time (like CPU, network send rate, file 
%write rate, etc.) the program is paused if it tries to over use the resource 
%so that the performance of the system does not suffer.   After a suitable 
%amount of time, the program is restarted.    If the resource is not renewable 
%(like the number of open files, memory use, disk used, etc.) then either the 
%function raises an exception or the program is killed.   This meets our
%design goal of preventing the 
%program from negatively impacting the performance of other programs running 
%on the same system.
%


\subsection{Node Manager}
\label{sec-nodemanager}
While useful in itself, the sandbox is part of a larger ecosystem.
The sandbox isolates a specific running program on a host computer,
but does not address how that program is started, which programs are
run, and who has permission to run a program. Such functionality is
provided by the node manager, which manages sandboxed running programs
as part of what we call \emph{vessels}. The node manager stores
information about the vessels it controls and allows vessels to be
started, stopped, combined, split, and changed.


\subsubsection{Vessels}

A vessel is a controlled environment for running code (implemented
using the Seattle sandbox).  Intuitively, a vessel includes the
program's sandbox and the node manager state (such as the resources
and restrictions assigned to the program).  Vessels have well defined
boundaries that prevent them from interfering with one another (for
example, different vessels have their own disjoint set of network
ports).  Each vessel has associated with it a restrictions file, a
stop file, and a log.  The restrictions file lists what the vessel can
and cannot do (such as the network ports that can be used) along with
the amount of each resource the vessel may use.  The stop file enables
the node manager to stop the vessel (by creating a file with that
name). The log is a circular buffer written by the vessel to
communicate useful information to the vessel owner. The log helps the
programmer to diagnose failures and to capture program state for
off-line analysis.

% (who has the initial set of resources since they installed
% the software)
A common scenario is for a student to obtain a vessel from their
instructor. The student then decides the program they want to run in
their vessel. To do this, she directs the experiment manager to
install the program on a node. The experiment manager uploads the
program to the student's vessel (along with any data files) and
executes the program in the vessel. The student also can easily perform
this action on groups of vessels spread across many different nodes.
The student can then monitor the status of her program by looking at a 
status indicator provided by querying the node manager (coarse-grained 
monitoring) or by downloading information 
about the program from its circular log buffer (fine-grained monitoring).  
The user can also stop the vessel while retaining all of the state so that 
she can examine data files and logs.   

In a more complex example an instructor splits a single vessel on a
node into multiple vessels, and
assigns each vessel to a student
in the class. Vessels may also be combined for flexibility. For example,
the students may be allowed to work in groups. Once the groups are
formed, some of the students may decide to combine their vessels so as
to get more resources in a single vessel.

% ib: do we need to describe offcut resources? seems too detailed.
% jac: commented out
%
%When a vessel is split, the quantity of resources isn't exactly
%divided between the two new vessels because of inherent overhead
%inherent in managing a vessel. To capture this effect, there is a
%resource specification for offcut resources.  Whenever a vessel is
%split, an amount of resources equal to the offcut resources is lost.
%Whenever two vessels are joined, the vessels gain resources equal to
%the offcut resources.

\subsection{Locating Seattle Nodes}
\label{sec-locationservice}

It is important to note that there is nothing parallel or distributed
about the node manager. The node manager only manages the vessels on
the local system. To facilitate global location of resources, the node
manager inserts a key/value pair into two different public DHTs
(OpenDHT~\cite{Rhea_SIGCOMM_2005}) every five minutes. The inserted key is
the owner's public key and the value is the local computer's IP
address. This allows a user to lookup their public key to find
the nodes with vessels they control without needing to keep track of
these nodes on their own.

\subsection{Experiment Manager}
\label{sec-experimentmanager}

The experiment manager is the main tool in the toolkit that students use
to interact with Seattle. The experiment manager
transparently handles discovery of vessels the user controls by
querying the DHT, and communicates with remote node managers to perform
actions on the user's behalf.

The experiment manager provides the user with a shell interface
(similar to PLuSH~\cite{Albrecht_LISA_2007}) in which the user can
issue commands. For example, users can install software in their
vessels -- the experiment manager uploads a program into the vessels
the user specifies. Users can also start and stop vessels, or report
on the status of a vessel. A vessel's status can be fresh (has never
run a program), started (is running a program), stopped (was requested
to stop), or terminated (terminated due to a normal exit or an unhandled 
program exception). When a vessel has failed
(perhaps due to a bug in the student's code), exception information
with a stack trace of the fault is logged. The student can use
the experiment manager to find the failed program instances to inspect
their logs or to see an exception's stack trace.



\section{Deployment}
\label{sec-deployment}

Since it is unclear what the future of cloud computing will be, we are
interested in providing students with the most diverse set of
resources possible. Seattle runs on Linux, FreeBSD, Mac OS-X, XO (one
laptop per child)\footnote[1]{Some threading libraries and operating
  systems do not provide accurate CPU and/or memory information. As a
  result, certain resources cannot be effectively sandboxed}, and
Windows platforms. Seattle also runs on mobile devices like the
iPhone\footnotemark[1] (if jail broken) and Nokia N800\footnotemark[1].
We are interested in adding support for other platforms as users
express interest in running Seattle on new platforms.

In addition to platform diversity, network connection diversity is
also important. Seattle is already widely deployed at universities
around the world. An instance of Seattle is running on each 
PlanetLab~\cite{PlanetLab} node, giving a presence on close to one thousand
nodes at hundreds of universities.  

PlanetLab provides access to computers at a large number of well
connected locations, although most only have two computers per site
available. As we are now publicly releasing Seattle for educational
use, we expect to see an increase in resource diversity.  We
anticipate that universities will install Seattle on most of the
computers in their lab. Such deployments will provide a distributed
environment that effectively emulates cluster computing. We also
anticipate that many students will load Seattle on their home
computers. This will allow for emulation of peer-to-peer computing as
home computers will typically have connectivity characteristics
representative of the average Internet user.

We believe that the diversity of platforms and network connections
enables a wide range of pedagogical uses. Students can experiment with
cluster computing, grid computing, peer-to-peer, and cloud computing
by simply varying where their program is deployed. Additionally, the
same program will run in any of these environments. Of course, the
characteristics of the environment will determine the efficiency and
scalability of the employed distributed algorithm --- a crucial
distributed systems lesson for students.


\section{Seattle in the Classroom}
\label{sec-assignments}

This section describes educational resources available to the student and 
instructor.   The resources are divided into two areas.   There is a student
portal which contains resources for students who are learning the basics of
Seattle.   There is also an educator portal that contains resources for 
educators to help them use Seattle in the classroom.   

\subsection{Student Resources}
To aid students learning Seattle, we provide a student portal with
tutorials describing how to program and use Seattle, documentation for the API,
a resources and restrictions guide, and other documentation.   The tutorials 
provide code snippets demonstrating the API, and explains how to use the 
experiment manager to perform different tasks.
Our experience has shown that students can quickly learn the Python
programming language~\cite{611980, Cappos_WORLDS_2005} and quickly learn
to program in Seattle.   Our experience is that undergraduates who had no 
previous networking experience can implement programs like overlay 
multicast and TCP forwarding with a few hours of effort after completing 
the tutorial.
% JAC: Should this be removed? Is mentioning ease of learning a distraction?
% ib: i think this should stay, an important characteristic for teaching with Seattle

To illustrate how easy it is to program Seattle, 
A popular first project for networking students is the Echo
client/server. The two \emph{complete} Seattle programs are both
concise and simple. The echo client is just 6 lines of code:

{\scriptsize
\begin{verbatim}
# Handle an incoming message
def got_reply(srcip, srcport, mess, ch):
  print 'received:',mess,"from",srcip,srcport

if callfunc == 'initialize':
  # when a message arrives on my IP, port 43210, 
  # start an event to call the function 'got_reply'
  recvmess(getmyip(), 43210, got_reply)  
  # send a hello message to my IP, port 54321
  sendmess(getmyip(), 54321, 'hello', getmyip(), 43210)  
  # exit in one second
  settimer(1,exitall,())
\end{verbatim}
}

The echo server consists of just 4 lines of code:

{\scriptsize
\begin{verbatim}
# Handle an incoming message
def got_message(srcip, srcport, mess, ch):
  sendmess(srcip,srcport,mess)

if callfunc == 'initialize':
  recvmess(getmyip(), 54321, got_message)  
\end{verbatim}
}


Furthermore, a first project for many students in distributed systems is to 
measure the connectivity characteristics between pairs of
computers~\cite{AllPairsPing}.  The Appendix lists a complete 30 line Seattle
program that performs an all-pairs-ping and displays its results in a webpage 
when contacted.


\subsection{Educator Resources}

To aid instructors in integrating Seattle into their existing curriculum, 
we also provide an educator portal.   The first item on the educator 
portal is a description of instructor experiences with Seattle.   After
Seattle has been used, we have asked the students to fill out a survey 
outlining how they felt the platform impacted their experience.   We recorded
the results on the web page.

Besides information about experiences with Seattle, the educator portal also 
contains course materials such as handouts and example assignments.   
One special-purpose assignment called the TakeHomeAssignment requires 
no programming and takes about 1 hour to do.   The purpose of the 
TakeHomeAssignment is to show the student or instructor the practical effects 
of non-transitive connectivity and NATs on the Internet today, while 
introducing them to Seattle.

In addition, the educator portal provides a collection of ready-to-go 
programming assignments for use with Seattle.   The assignments include implementing a 
reliable protocol on top of UDP, performing overlay routing using link-state 
routing on the Internet, building application level services like a webserver, and 
understand layering of services by creating a chat server that operates 
over HTTP.   

The educator portal also provides assignment ideas that are appropriate 
for course long projects or graduate assignments.   For example, students 
may implement a DHT (such as Chord~\cite{Stoica_SIGCOMM_2001}) to better 
understand non-transitive connectivity.   A simple implementation will work
well on LAN environments but will fail horribly on the Internet due to 
non-transitive connectivity and other network effects.   After measuring
these effects and then understanding the reason behind Chord's poor 
performance, the students can discuss solutions to these
problems. Students can then implement and test these solutions to achieve
better performance and reliability. 
This assignment emphasizes good software engineering practices (since code is
reused), that test and deployment environments may differ significantly, 
and encourages students to come up with unique solutions to the problem yet 
is easily evaluated using a small set of metrics.



\section{Related Work}
\label{sec-relatedwork}

%As the
%future of cloud computing is far from clear, Seattle benefits courses
%varying in their coverage of parallel processing and distributed
%computing~\cite{SIGCSE98}.

%Seattle's Python-based interface is simple to learn, and easy to
%master. Yet, it retains much power in the hands of the programmer.
%Seattle can provide a fun experience for students without them having
%to master advanced networking API features nor organize the
%distribution of their code to other platforms or having to organize a
%testbed of machines.

There are many different cloud computing platforms in use today.
Amazon runs a cloud of RedHat servers to provide computing
resources~\cite{AmazonEC2},
which %and a storage back end~\cite{AmazonS3}.
are similar in purpose to Seattle but provide a virtual machine
instead of a programming language instance. This leads to better
performance, but is not flexible to support donated resources and is
not free. Their storage back end is functionally similar to the global
data store proposed as an assignment on Seattle's educator portal.

Microsoft proposes a software plus services~\cite{MicrosoftMemo}
architecture where the cloud is used as an auxiliary to enhance the
capabilities of local software. While Microsoft has announced the
pending availability of a developer toolkit, to the best of our
knowledge it is not available at this time. % for further comparison.

Google provides a cloud computing-like infrastructure with
AppEngine~\cite{GoogleAppEngine}, which executes programs written in a
constrained version of the Python language and supports high level
abstractions (like a global database). While useful for building
locality-oblivious web applications, its transparent handling of
scalability and locality makes it unsuitable for teaching these
fundamental distributed systems topics.

IBM has announced plans for a cloud computing product called ``Blue
Cloud''~\cite{IBMBlueCloud}, which supports OS images using Xen and
PowerVM Linux. Blue Cloud also supports Hadoop~\cite{Hadoop} for
MapReduce-type query processing and is intended to support high
performance computing. %so that organizations can migrate supercomputer
%tasks to Blue Cloud.
Hadoop has also been used to teach cluster computing for large data
processing~\cite{SIGCSE08}. The MapReduce~\cite{MapReduce} paradigm
used by Hadoop simplifies distributed data processing. However, this
simplifying abstraction also limits the scope of systems concepts that
may be taught with Hadoop. We believe that a more general platform
should be used to teach the system concepts that power Hadoop's
implementation. These concepts may then be applied more broadly to
other distributed computing abstractions, and cloud computing more
generally.

In addition to cloud computing, there are a variety of grid computing
and volunteer computing platforms. Globus~\cite{Globus} is a popular
Grid toolkit, which has been used to build a variety of
service-oriented applications. BOINC~\cite{BOINC} is a volunteer
computing platform supporting the SETI@Home and Folding@Home projects.
BOINC leverages donated CPU cycles for computation, particularly spare
resources on home machines. %To BOINC, getting 1\% of the
%time on 1000 computers is similar to getting 100\% on 10 computers
%(for similar computers). With Seattle, it is much more useful to get
%1\% on 1000 computers because the research and educational programs
%are trying to understand how to write scalable programs that run on
%many computers at the same time.
Globus and BOINC both target distributed computation and strive to
hide locality and similar information from the programmer. Seattle is
lighter-weight software that exposes locality and is therefore suited
for students in distributed systems courses. Seattle also comes with a
widely accessible ready-to-use platform of thousands of machines. We
are not aware of any Globus- or BOINC-powered testbeds available for
educational use.

%The problems and
%programs become interesting when there are thousands of computers.

%% Also most volunteer computing platforms like BOINC use CPU cycles only
%% when the system would be idle so as to avoid interfering with the
%% user.  BOINC typically uses almost all of the CPU when it's idle.
%% With Seattle, it is more desirable to constantly have a small
%% percentage of the resources (to allow a more constant set of resources
%% to be used over a longer time period).

Limited compute resources have been a key constraint in teaching
distributed systems~\cite{SIGCSE94}. Seattle is architected to do this
safely and efficiently. % across a variety of architectures.
Recent efforts engaged undergraduates in distributed computing with
simple-to-use platforms designed for cluster
computing. %~\cite{SIGCSE07}.
The DCEZ platform offers a simple interface that students can use
without any prior knowledge of distributed
computing~\cite{SIGCSE07}. %With minimal assistance, students were
%shown to install and use a small cluster with DCEZ within 5 to 10
%minutes.
We have likewise endeavored to make Seattle simple to use, but
target teaching of distributed systems issues that arise at Internet
scales. Lastly, because Seattle runs on a variety of embedded
platforms with limited resources, such as cellular phones and PDAs,
our platform complements prior work on platforms for teaching
ubiquitous computing~\cite{SIGCSE03}.


\section{Conclusion and Future Work}
\label{sec-conclusion}
This work presents the educational networking platform Seattle. Seattle is a
free, portable, and lightweight platform using donated computational
resources. Seattle enables students to learn the concepts of networking and
distributed systems on computers spread around the Internet.   Seattle
can also emulate cloud computing, peer-to-peer computing, and cluster 
computing within a simple framework.   Computers running Seattle are protected 
from malicious and misbehaving code, making it safe to contribute
resources from multi-use computers.
Seattle has resources available for students to use on about a thousand 
computers worldwide.

We are currently working to extend Seattle to improve Seattle's NAT 
traversal and to provide better aggregate restrictions on Seattle traffic.


%ACKNOWLEDGMENTS are optional
%\section{Acknowledgments}


%
% The following two commands are all you need in the
% initial runs of your .tex file to
% produce the bibliography for the citations in your paper.

{%\small
\bibliographystyle{abbrv}
\bibliography{bibdata}}  % paper.bib is the name of the Bibliography in this case

%% THIS IS SIGPROC-SP.TEX - VERSION 2.9
% WORKS WITH V3.0SP OF ACM_PROC_ARTICLE-SP.CLS
% MARCH 2007
%
% It is an example file showing how to use the 'acm_proc_article-sp.cls' V3.0SP
% LaTeX2e document class file for Conference Proceedings submissions.
% ----------------------------------------------------------------------------------------------------------------
% This .tex file (and associated .cls V3.0SP) *DOES NOT* produce:
%       1) The Permission Statement
%       2) The Conference (location) Info information
%       3) The Copyright Line with ACM data
%       4) Page numbering
% ---------------------------------------------------------------------------------------------------------------
% It is an example which *does* use the .bib file (from which the .bbl file
% is produced).
% REMEMBER HOWEVER: After having produced the .bbl file,
% and prior to final submission,
% you need to 'insert'  your .bbl file into your source .tex file so as to provide
% ONE 'self-contained' source file.
%
% Questions regarding SIGS should be sent to
% Adrienne Griscti ---> griscti@acm.org
%
% Questions/suggestions regarding the guidelines, .tex and .cls files, etc. to
% Gerald Murray ---> murray@acm.org
%
% For tracking purposes - this is V2.9SP - MARCH 2007

\documentclass{sig-alternate}
\usepackage{verbatim}
\usepackage{url}

%\pagenumbering{arabic}

%-- Begin patch area for accents in 'Author Block' area - may be needed by some authors / but not all
\DeclareFixedFont{\auacc}{OT1}{phv}{m}{n}{12}   % Needed for "Author Block" accents - Patch by Gerry 3/21/07
\DeclareFixedFont{\afacc}{OT1}{phv}{m}{n}{10}   % Needed for "Author Block" accents in the affiliation/address line - Patch by Gerry 3/21/07
%--
\newtheorem{remark}{Remark}

% I'm using this so I can easily mark parts that are moved to the TR...
\newcommand{\movetotr}[1]{{}}

\begin{document}
\conferenceinfo{SIGCSE'09,} {March 3--7, 2009, Chattanooga, Tennessee, USA.} 

\CopyrightYear{2009}

\crdata{978-1-60558-183-5/09/03} 

\title{Seattle: A Platform for Educational Cloud Computing\thanks{This work 
was partially supported by NSF Grant CNS-0834243}
}
%
% You need the command \numberofauthors to handle the 'placement
% and alignment' of the authors beneath the title.
%
% For aesthetic reasons, we recommend 'three authors at a time'
% i.e. three 'name/affiliation blocks' be placed beneath the title.
%
% NOTE: You are NOT restricted in how many 'rows' of
% "name/affiliations" may appear. We just ask that you restrict
% the number of 'columns' to three.
%
% Because of the available 'opening page real-estate'
% we ask you to refrain from putting more than six authors
% (two rows with three columns) beneath the article title.
% More than six makes the first-page appear very cluttered indeed.
%
% Use the \alignauthor commands to handle the names
% and affiliations for an 'aesthetic maximum' of six authors.
% Add names, affiliations, addresses for
% the seventh etc. author(s) as the argument for the
% \additionalauthors command.
% These 'additional authors' will be output/set for you
% without further effort on your part as the last section in
% the body of your article BEFORE References or any Appendices.

%\numberofauthors{5} %  in this sample file, there are a *total*

% of EIGHT authors. SIX appear on the 'first-page' (for formatting
% reasons) and the remaining two appear in the \additionalauthors section.
%
\author{
%Author Names Removed for Anonymous Submission
Justin Cappos~~~~Ivan Beschastnikh~~~~Arvind Krishnamurthy~~~~Tom Anderson\\[1ex]
       \affaddr{Department of Computer Science and Engineering, University of Washington}\\
       \affaddr{Seattle, WA 98105, U.S.A.}\\
       \email{\{justinc, ivan, arvind, tom\}@cs.washington.edu}
}
\date{}
%\hyphenation{Com-put-er-Comm-uni-ca-tion Comm-uni-ca-tion}

\maketitle
\abstract

%As many organizations outsource hardware and maintainence and instead
%focus on software, % IB: do we need give a reason for the transition?

Cloud computing is rapidly increasing in popularity. Companies such as
RedHat, Microsoft, Amazon, Google, and IBM are increasingly funding
cloud computing infrastructure and research, making it important for
students to gain the necessary skills to work with cloud-based
resources. This paper presents a free, educational research platform
called Seattle that is community-driven, a common denominator for
diverse platform types, and is broadly deployed.

Seattle is community-driven --- universities donate available compute
resources on multi-user machines to the platform. These donations can
come from systems with a wide variety of operating systems and
architectures, removing the need for a dedicated infrastructure.
% to support Seattle.

Seattle is also surprisingly flexible and supports a variety of
pedagogical uses because as a platform it represents a common denominator 
for cloud computing, grid computing, peer-to-peer networking,
distributed systems, and networking. Seattle programs are portable.
Students' code can run across different operating systems and
architectures without change, while the Seattle programming language
is expressive enough for experimentation at a fine-grained level. Our
current deployment of Seattle consists of about one thousand computers
that are distributed around the world. We invite the computer science
education community to employ Seattle in their courses.

% mention planetlab

% arvind's comments:
%Community-driven (resources are provided by the community).
%
%Common denominator for wide range of computing types
%
%Free
%
%Broad deployment (many hundreds of systems)

%Cloud computing is rapidly increasing in popularity as many organizations
%outsource hardware and maintainence and instead focus on software.   With
%major industry players like RedHat, Microsoft, Amazon, Google, and IBM
%investing heavily in cloud computing infrastructure and research, it is 
%important that students gain the skills they need to work with cloud-based 
%resources.   
%We have built a cloud computing platform called Seattle for the
%purpose of allowing students to experiment with cloud computing in a safe,
%portable, and cost-free environment.   Computers that run Seattle are protected
%from buggy and malicious programs by a sandboxing mechanism that provides 
%security and performance isolation.  Students that use Seattle can write 
%programs that use resources on around one thousand computers running at 
%hundreds of universities distributed around the world.   An instructor can
%use Seattle to demonstrate the difference between cluster computing 
%(possibly remote computers with low inter-computer latency), cloud 
%computing (collaborative computers where at least some systems have low
%latency to the source), peer-to-peer computing (randomly distributed 
%computers), and other in-between scenarios.

% A category with the (minimum) three required fields
\category{K.3.2}{Computer and Information Science Education}{Computer science education}
\category{C.2.4}{Computer-Communi-cation Networks}{Distributed Systems}[client/server, distributed applications]
\category{C.4}{Computer Systems Organization}{Performance of Systems}[design studies, measurement techniques]

%A category including the fourth, optional field follows...\category{D.2.8}{Software Engineering}{Metrics}[complexity measures, performance measures]
\terms{Experimentation, Measurement, Performance, Security}

\keywords{Cloud computing, peer-to-peer computing, cluster computing, distributed computing}
\section{Introduction}

Cloud computing is rapidly increasing in popularity as many organizations
outsource hardware and maintenance and instead focus on 
software~\cite{AmazonEC2, AmazonS3, GoogleAppEngine, GoogleDocs, HPCloud, 
IBMBlueCloud, MicrosoftMemo}.   
%Amazon
%has built an infrastructure for Elastic Cloud Computing (EC2)~\cite{AmazonEC2} 
%on top of RedHat servers.   Google has provided a programming API and platform
%called AppEngine~\cite{GoogleAppEngine} that transparently provides scalability
%while hiding locality information from applications.   Google also provides
%applications like Google Docs~\cite{GoogleDocs} which are built using cloud
%computing principles.  Other major industry players like 
%Microsoft~\cite{MicrosoftMemo} and IBM~\cite{IBMBlueCloud} have announced their
%own cloud computing platforms and HP, Intel, and Yahoo!~\cite{HPCloud} have 
%announced a joint pushes for cloud computing research.   
%%%%%
%%%% An attempt where I talk about client software as well as cloud differences
%However, despite all of the focus on cloud computing, there
%is a lot of disparity in industry in terms of what cloud computing 
%really means.   RedHat / Amazon's EC2~\cite{AmazonEC2} provides cloud
%computing as a collection of Linux boxes along with
%storage functionality~\cite{AmazonS3} while leaving the client interaction
%with the cloud up to the developer.   Google provides automatic scaling of
%cloud computing programs written to a custom programming 
%API~\cite{GoogleAppEngine} and tends to have clients access cloud resources 
%using their web browser~\cite{GoogleDocs}.   Microsoft views it as a 
%virtualization layer between the hardware and OS and is releasing 
%developer toolkits to be used in a ``software plus service'' architecture where
%the client runs custom cloud software on their local 
%machine~\cite{MicrosoftMemo}.
%%%%%
However, despite the attention, there is a lot of disparity in what
cloud computing means. RedHat / Amazon's EC2~\cite{AmazonEC2} provides
cloud computing as a collection of Linux machines with storage
functionality~\cite{AmazonS3}. Google's platform for cloud computing
hides locality and scalability issues from the programmer who writes
programs to a custom programming API~\cite{GoogleAppEngine}. Microsoft
views it as a virtualization layer between the hardware and the OS and
is releasing a developer toolkit for providing the user with
``software plus service.''~\cite{MicrosoftMemo}

We provide an educational platform called Seattle that is a common
denominator for a wide range of these definitions. Seattle's simple to
learn programming language, a subset of the Python language, is
expressive enough to allow students to build algorithms for
inter-machine interaction (like a global store or a DHT). As a result,
Seattle is useful in many pedagogical contexts ranging from courses in
cloud computing, networking, and distributed systems, to parallel
programming, grid computing, and peer-to-peer computing.

Seattle is a community-driven effort that depends on resources donated
by users of the software (and as such is free to use). A user can
install Seattle onto their personal computer to enable Seattle
programs to run using a portion of the computer's resources. Seattle
programs are sandboxed and securely isolated from other programs
running on the same computer. Seattle provides hard resource
guarantees that an erroneous or malicious program cannot circumvent.

In addition, Seattle runs on a variety of different operating systems
and architectures including Windows, Mac OS-X, Linux, FreeBSD, and even
portable devices like Nokia N800s and jail broken iPhones. Code written
for Seattle is automatically (and transparently) portable to different
architectures and runs the same across all systems.

Seattle has a preexisting base of installed computers and is already
widely deployed on almost one thousand computers that are spread
across hundreds of universities worldwide. Seattle users can run their
programs on computers spanning the Internet -- a feature that is
currently being used by several classes at major universities.

%While Seattle is useful as a cloud computing platform, there are
%other important pedagogical concepts that Seattle can illustrate.
%The first concept is that the characteristics of links between computing 
%resources 
%matters.  In other words, two computers in a data center behave differently 
%than two computers connected by a trans-atlantic cable.   The second concept is
%that the grouping of resources is important.   It is possible to efficiently 
%solve different sets of problems with 100\% of one CPU than 10\% of 10 
%computers.   The third concept is that failure handling is important.   In 
%globally distributed applications, communications failures due to routing
%black holes or DNS issues are commonplace~\cite{Katz_NSDI_2008}.   Applications
%that seek to provide high uptime must handle failures well.

%\subsection{Map}
This paper describes the architecture of the Seattle cloud computing
platform (Section~\ref{sec-architecture}) including the
programming API (Section~\ref{sec-API}), 
the sandboxing mechanism (Section~\ref{sec-repy}), 
the control of sandboxes on a
host computer (Section~\ref{sec-nodemanager}), and the way in which
students manage their running programs
(Section~\ref{sec-experimentmanager}). Following this, we describe the
computational resources available to classes using Seattle and how we
expect this platform to grow in the future
(Section~\ref{sec-deployment}).  Next, we provide some example
assignments to demonstrate how Seattle can be used in courses
(Section~\ref{sec-assignments}). We then discuss related work
(Section~\ref{sec-relatedwork}) and conclude
(Section~\ref{sec-conclusion}).




\section{Architecture}
\label{sec-architecture}

To use Seattle, the instructor creates an account on our website and obtains
an installer.   The machines that run the installer (such as computers in 
the universities computer lab) donate resources that are credited to the 
instructor.   The instructor can then obtain resources on machines around the 
world.   As of December 1st, 2008 the current policy is that for each donation,
the instructor receives resources on ten other computers.   However, the 
instructor can delegate those resources either directly to students or
to TAs who do more fine-grained delegation.   Students and TAs download
a toolkit and then experiment with their resources.

Seattle's architecture is comprised of several components. At the
lowest level the \emph{sandbox} component guarantees security and
resource control for an individual program. Programs are written to
the Seattle API in a subset of the Python programming language. This
API provides portable access to low level operations (like opening
files and sending messages). At a higher layer, the \emph{node
  manager} determines which sandboxed programs get to run on the local
computer. A public key infrastructure is used to authorize control
over sandboxed programs. Lastly, the \emph{experiment manager} lets
students control their program instances across computers.

\subsection{Seattle API}
\label{sec-API}

Seattle provides a programming API for low-level operations (like
writing to files or sending network messages) and maintains program
portability using an abstraction layer. Platform specific code below
this abstraction layer handles non-portable operations enabling
unmodified programs to run on a wide variety of
platforms. % This also allows the majority of the sandbox to be reused
%across different architectures.


The API consists of five categories: file, network, timer, locking,
and miscellaneous. The file API calls enable limited access to the
local computer's persistent storage (interacting only with files in a 
single directory). 
%the calls to
%create, delete, open, read and write sandboxed files, as well as calls
%to list directory contents in the sandbox. 
The network API calls
provide the local IP address, perform a DNS lookup, enable sending and
receiving of UDP messages, and managing of and communicating over TCP
connections. The timer API calls enable the programmer to put the
current thread to sleep and to schedule functions to be called at
later times. For example, the programmer can register an event to
periodically send a heartbeat message to another computer. The locking
functions allow the programmer to handle concurrency in their program
(as common state may be accessed and modified by multiple threads at
the same time). The miscellaneous API calls allow the programmer to
exit the program, to generate random numbers, and to provide the
amount of time the program has been running.

% initiation/termination of and listening for TCP connections

\subsection{The Sandbox}
\label{sec-repy}

The sandbox's primary goal is to securely execute user code. There are
two aspects to this --- preventing insecure actions and limiting
resource consumption. To prevent insecure actions the sandbox hooks into 
the Python parser and reads the program's parse tree. Only actions that
the sandbox can verify as safe may execute.

To control resource consumption on the host the sandbox interposes on
all API calls made by a program. The sandbox monitors the overall use
of resources like CPU, memory, and disk space to ensure the program
does not exceed its bounds. Each API call that uses a monitored
resource is evaluated before being granted or denied the resource. The
sandbox also restricts the rate at which API calls are performed.

The restrictions and resource limits of the sandbox are configurable and
may restrict different programs in different ways. For example, one 
program may only be
allowed to receive UDP packets on port 11111, while another program
may be restricted to receiving UDP packets on port 22222. This enables
multiple sandboxes on the same computer to host programs controlled by
different users.

%We provide an interface where the program can execute unsafe operations (such 
%as open files, send network traffic, etc.).   Our sandbox verifies that the 
%program is performing a safe operation (like opening a file in the program's 
%sandbox) instead of a malicious action (like opening the user's credit card 
%information).   This is done by fine grained checks on the arguments passed 
%to individual calls.   Once again we err on the side of caution (for example, 
%restricting the names of files the program can open) to prevent the program 
%from escaping the sandbox.
%
%However, not every safe action should be allowed by every program.   It may 
%be that two programs would like to run on the same system and each program 
%has been allocated its own port.   The programs should not be able to use 
%the other program's port or other ports on the system.   To prevent this 
%repy has per-program fine-grained restrictions that control the use of 
%calls.   Program A can be restricted to port 12345 while Program B is 
%restricted to port 54321.   Similarly, a Program C that does not need 
%to write files to the disk can be prevented from using file operations all 
%together!   This meets our design goal of a multi-user programming environment
%since students are isolated from each other.
%
%
%\subsubsection{Resource Limits}
%\label{sec-resourcelimits}
%
%Another important type of isolation is resource isolation.   One program that 
%runs on a computer should not be able to consume enough resources to 
%impact the execution of other programs or the host computer.   To this end, 
%repy limits resource consumption through several mechanisms.   For 
%resources that renew themselves over time (like CPU, network send rate, file 
%write rate, etc.) the program is paused if it tries to over use the resource 
%so that the performance of the system does not suffer.   After a suitable 
%amount of time, the program is restarted.    If the resource is not renewable 
%(like the number of open files, memory use, disk used, etc.) then either the 
%function raises an exception or the program is killed.   This meets our
%design goal of preventing the 
%program from negatively impacting the performance of other programs running 
%on the same system.
%


\subsection{Node Manager}
\label{sec-nodemanager}
While useful in itself, the sandbox is part of a larger ecosystem.
The sandbox isolates a specific running program on a host computer,
but does not address how that program is started, which programs are
run, and who has permission to run a program. Such functionality is
provided by the node manager, which manages sandboxed running programs
as part of what we call \emph{vessels}. The node manager stores
information about the vessels it controls and allows vessels to be
started, stopped, combined, split, and changed.


\subsubsection{Vessels}

A vessel is a controlled environment for running code (implemented
using the Seattle sandbox).  Intuitively, a vessel includes the
program's sandbox and the node manager state (such as the resources
and restrictions assigned to the program).  Vessels have well defined
boundaries that prevent them from interfering with one another (for
example, different vessels have their own disjoint set of network
ports).  Each vessel has associated with it a restrictions file, a
stop file, and a log.  The restrictions file lists what the vessel can
and cannot do (such as the network ports that can be used) along with
the amount of each resource the vessel may use.  The stop file enables
the node manager to stop the vessel (by creating a file with that
name). The log is a circular buffer written by the vessel to
communicate useful information to the vessel owner. The log helps the
programmer to diagnose failures and to capture program state for
off-line analysis.

% (who has the initial set of resources since they installed
% the software)
A common scenario is for a student to obtain a vessel from their
instructor. The student then decides the program they want to run in
their vessel. To do this, she directs the experiment manager to
install the program on a node. The experiment manager uploads the
program to the student's vessel (along with any data files) and
executes the program in the vessel. The student also can easily perform
this action on groups of vessels spread across many different nodes.
The student can then monitor the status of her program by looking at a 
status indicator provided by querying the node manager (coarse-grained 
monitoring) or by downloading information 
about the program from its circular log buffer (fine-grained monitoring).  
The user can also stop the vessel while retaining all of the state so that 
she can examine data files and logs.   

In a more complex example an instructor splits a single vessel on a
node into multiple vessels, and
assigns each vessel to a student
in the class. Vessels may also be combined for flexibility. For example,
the students may be allowed to work in groups. Once the groups are
formed, some of the students may decide to combine their vessels so as
to get more resources in a single vessel.

% ib: do we need to describe offcut resources? seems too detailed.
% jac: commented out
%
%When a vessel is split, the quantity of resources isn't exactly
%divided between the two new vessels because of inherent overhead
%inherent in managing a vessel. To capture this effect, there is a
%resource specification for offcut resources.  Whenever a vessel is
%split, an amount of resources equal to the offcut resources is lost.
%Whenever two vessels are joined, the vessels gain resources equal to
%the offcut resources.

\subsection{Locating Seattle Nodes}
\label{sec-locationservice}

It is important to note that there is nothing parallel or distributed
about the node manager. The node manager only manages the vessels on
the local system. To facilitate global location of resources, the node
manager inserts a key/value pair into two different public DHTs
(OpenDHT~\cite{Rhea_SIGCOMM_2005}) every five minutes. The inserted key is
the owner's public key and the value is the local computer's IP
address. This allows a user to lookup their public key to find
the nodes with vessels they control without needing to keep track of
these nodes on their own.

\subsection{Experiment Manager}
\label{sec-experimentmanager}

The experiment manager is the main tool in the toolkit that students use
to interact with Seattle. The experiment manager
transparently handles discovery of vessels the user controls by
querying the DHT, and communicates with remote node managers to perform
actions on the user's behalf.

The experiment manager provides the user with a shell interface
(similar to PLuSH~\cite{Albrecht_LISA_2007}) in which the user can
issue commands. For example, users can install software in their
vessels -- the experiment manager uploads a program into the vessels
the user specifies. Users can also start and stop vessels, or report
on the status of a vessel. A vessel's status can be fresh (has never
run a program), started (is running a program), stopped (was requested
to stop), or terminated (terminated due to a normal exit or an unhandled 
program exception). When a vessel has failed
(perhaps due to a bug in the student's code), exception information
with a stack trace of the fault is logged. The student can use
the experiment manager to find the failed program instances to inspect
their logs or to see an exception's stack trace.



\section{Deployment}
\label{sec-deployment}

Since it is unclear what the future of cloud computing will be, we are
interested in providing students with the most diverse set of
resources possible. Seattle runs on Linux, FreeBSD, Mac OS-X, XO (one
laptop per child)\footnote[1]{Some threading libraries and operating
  systems do not provide accurate CPU and/or memory information. As a
  result, certain resources cannot be effectively sandboxed}, and
Windows platforms. Seattle also runs on mobile devices like the
iPhone\footnotemark[1] (if jail broken) and Nokia N800\footnotemark[1].
We are interested in adding support for other platforms as users
express interest in running Seattle on new platforms.

In addition to platform diversity, network connection diversity is
also important. Seattle is already widely deployed at universities
around the world. An instance of Seattle is running on each 
PlanetLab~\cite{PlanetLab} node, giving a presence on close to one thousand
nodes at hundreds of universities.  

PlanetLab provides access to computers at a large number of well
connected locations, although most only have two computers per site
available. As we are now publicly releasing Seattle for educational
use, we expect to see an increase in resource diversity.  We
anticipate that universities will install Seattle on most of the
computers in their lab. Such deployments will provide a distributed
environment that effectively emulates cluster computing. We also
anticipate that many students will load Seattle on their home
computers. This will allow for emulation of peer-to-peer computing as
home computers will typically have connectivity characteristics
representative of the average Internet user.

We believe that the diversity of platforms and network connections
enables a wide range of pedagogical uses. Students can experiment with
cluster computing, grid computing, peer-to-peer, and cloud computing
by simply varying where their program is deployed. Additionally, the
same program will run in any of these environments. Of course, the
characteristics of the environment will determine the efficiency and
scalability of the employed distributed algorithm --- a crucial
distributed systems lesson for students.


\section{Seattle in the Classroom}
\label{sec-assignments}

This section describes educational resources available to the student and 
instructor.   The resources are divided into two areas.   There is a student
portal which contains resources for students who are learning the basics of
Seattle.   There is also an educator portal that contains resources for 
educators to help them use Seattle in the classroom.   

\subsection{Student Resources}
To aid students learning Seattle, we provide a student portal with
tutorials describing how to program and use Seattle, documentation for the API,
a resources and restrictions guide, and other documentation.   The tutorials 
provide code snippets demonstrating the API, and explains how to use the 
experiment manager to perform different tasks.
Our experience has shown that students can quickly learn the Python
programming language~\cite{611980, Cappos_WORLDS_2005} and quickly learn
to program in Seattle.   Our experience is that undergraduates who had no 
previous networking experience can implement programs like overlay 
multicast and TCP forwarding with a few hours of effort after completing 
the tutorial.
% JAC: Should this be removed? Is mentioning ease of learning a distraction?
% ib: i think this should stay, an important characteristic for teaching with Seattle

To illustrate how easy it is to program Seattle, 
A popular first project for networking students is the Echo
client/server. The two \emph{complete} Seattle programs are both
concise and simple. The echo client is just 6 lines of code:

{\scriptsize
\begin{verbatim}
# Handle an incoming message
def got_reply(srcip, srcport, mess, ch):
  print 'received:',mess,"from",srcip,srcport

if callfunc == 'initialize':
  # when a message arrives on my IP, port 43210, 
  # start an event to call the function 'got_reply'
  recvmess(getmyip(), 43210, got_reply)  
  # send a hello message to my IP, port 54321
  sendmess(getmyip(), 54321, 'hello', getmyip(), 43210)  
  # exit in one second
  settimer(1,exitall,())
\end{verbatim}
}

The echo server consists of just 4 lines of code:

{\scriptsize
\begin{verbatim}
# Handle an incoming message
def got_message(srcip, srcport, mess, ch):
  sendmess(srcip,srcport,mess)

if callfunc == 'initialize':
  recvmess(getmyip(), 54321, got_message)  
\end{verbatim}
}


Furthermore, a first project for many students in distributed systems is to 
measure the connectivity characteristics between pairs of
computers~\cite{AllPairsPing}.  The Appendix lists a complete 30 line Seattle
program that performs an all-pairs-ping and displays its results in a webpage 
when contacted.


\subsection{Educator Resources}

To aid instructors in integrating Seattle into their existing curriculum, 
we also provide an educator portal.   The first item on the educator 
portal is a description of instructor experiences with Seattle.   After
Seattle has been used, we have asked the students to fill out a survey 
outlining how they felt the platform impacted their experience.   We recorded
the results on the web page.

Besides information about experiences with Seattle, the educator portal also 
contains course materials such as handouts and example assignments.   
One special-purpose assignment called the TakeHomeAssignment requires 
no programming and takes about 1 hour to do.   The purpose of the 
TakeHomeAssignment is to show the student or instructor the practical effects 
of non-transitive connectivity and NATs on the Internet today, while 
introducing them to Seattle.

In addition, the educator portal provides a collection of ready-to-go 
programming assignments for use with Seattle.   The assignments include implementing a 
reliable protocol on top of UDP, performing overlay routing using link-state 
routing on the Internet, building application level services like a webserver, and 
understand layering of services by creating a chat server that operates 
over HTTP.   

The educator portal also provides assignment ideas that are appropriate 
for course long projects or graduate assignments.   For example, students 
may implement a DHT (such as Chord~\cite{Stoica_SIGCOMM_2001}) to better 
understand non-transitive connectivity.   A simple implementation will work
well on LAN environments but will fail horribly on the Internet due to 
non-transitive connectivity and other network effects.   After measuring
these effects and then understanding the reason behind Chord's poor 
performance, the students can discuss solutions to these
problems. Students can then implement and test these solutions to achieve
better performance and reliability. 
This assignment emphasizes good software engineering practices (since code is
reused), that test and deployment environments may differ significantly, 
and encourages students to come up with unique solutions to the problem yet 
is easily evaluated using a small set of metrics.



\section{Related Work}
\label{sec-relatedwork}

%As the
%future of cloud computing is far from clear, Seattle benefits courses
%varying in their coverage of parallel processing and distributed
%computing~\cite{SIGCSE98}.

%Seattle's Python-based interface is simple to learn, and easy to
%master. Yet, it retains much power in the hands of the programmer.
%Seattle can provide a fun experience for students without them having
%to master advanced networking API features nor organize the
%distribution of their code to other platforms or having to organize a
%testbed of machines.

There are many different cloud computing platforms in use today.
Amazon runs a cloud of RedHat servers to provide computing
resources~\cite{AmazonEC2},
which %and a storage back end~\cite{AmazonS3}.
are similar in purpose to Seattle but provide a virtual machine
instead of a programming language instance. This leads to better
performance, but is not flexible to support donated resources and is
not free. Their storage back end is functionally similar to the global
data store proposed as an assignment on Seattle's educator portal.

Microsoft proposes a software plus services~\cite{MicrosoftMemo}
architecture where the cloud is used as an auxiliary to enhance the
capabilities of local software. While Microsoft has announced the
pending availability of a developer toolkit, to the best of our
knowledge it is not available at this time. % for further comparison.

Google provides a cloud computing-like infrastructure with
AppEngine~\cite{GoogleAppEngine}, which executes programs written in a
constrained version of the Python language and supports high level
abstractions (like a global database). While useful for building
locality-oblivious web applications, its transparent handling of
scalability and locality makes it unsuitable for teaching these
fundamental distributed systems topics.

IBM has announced plans for a cloud computing product called ``Blue
Cloud''~\cite{IBMBlueCloud}, which supports OS images using Xen and
PowerVM Linux. Blue Cloud also supports Hadoop~\cite{Hadoop} for
MapReduce-type query processing and is intended to support high
performance computing. %so that organizations can migrate supercomputer
%tasks to Blue Cloud.
Hadoop has also been used to teach cluster computing for large data
processing~\cite{SIGCSE08}. The MapReduce~\cite{MapReduce} paradigm
used by Hadoop simplifies distributed data processing. However, this
simplifying abstraction also limits the scope of systems concepts that
may be taught with Hadoop. We believe that a more general platform
should be used to teach the system concepts that power Hadoop's
implementation. These concepts may then be applied more broadly to
other distributed computing abstractions, and cloud computing more
generally.

In addition to cloud computing, there are a variety of grid computing
and volunteer computing platforms. Globus~\cite{Globus} is a popular
Grid toolkit, which has been used to build a variety of
service-oriented applications. BOINC~\cite{BOINC} is a volunteer
computing platform supporting the SETI@Home and Folding@Home projects.
BOINC leverages donated CPU cycles for computation, particularly spare
resources on home machines. %To BOINC, getting 1\% of the
%time on 1000 computers is similar to getting 100\% on 10 computers
%(for similar computers). With Seattle, it is much more useful to get
%1\% on 1000 computers because the research and educational programs
%are trying to understand how to write scalable programs that run on
%many computers at the same time.
Globus and BOINC both target distributed computation and strive to
hide locality and similar information from the programmer. Seattle is
lighter-weight software that exposes locality and is therefore suited
for students in distributed systems courses. Seattle also comes with a
widely accessible ready-to-use platform of thousands of machines. We
are not aware of any Globus- or BOINC-powered testbeds available for
educational use.

%The problems and
%programs become interesting when there are thousands of computers.

%% Also most volunteer computing platforms like BOINC use CPU cycles only
%% when the system would be idle so as to avoid interfering with the
%% user.  BOINC typically uses almost all of the CPU when it's idle.
%% With Seattle, it is more desirable to constantly have a small
%% percentage of the resources (to allow a more constant set of resources
%% to be used over a longer time period).

Limited compute resources have been a key constraint in teaching
distributed systems~\cite{SIGCSE94}. Seattle is architected to do this
safely and efficiently. % across a variety of architectures.
Recent efforts engaged undergraduates in distributed computing with
simple-to-use platforms designed for cluster
computing. %~\cite{SIGCSE07}.
The DCEZ platform offers a simple interface that students can use
without any prior knowledge of distributed
computing~\cite{SIGCSE07}. %With minimal assistance, students were
%shown to install and use a small cluster with DCEZ within 5 to 10
%minutes.
We have likewise endeavored to make Seattle simple to use, but
target teaching of distributed systems issues that arise at Internet
scales. Lastly, because Seattle runs on a variety of embedded
platforms with limited resources, such as cellular phones and PDAs,
our platform complements prior work on platforms for teaching
ubiquitous computing~\cite{SIGCSE03}.


\section{Conclusion and Future Work}
\label{sec-conclusion}
This work presents the educational networking platform Seattle. Seattle is a
free, portable, and lightweight platform using donated computational
resources. Seattle enables students to learn the concepts of networking and
distributed systems on computers spread around the Internet.   Seattle
can also emulate cloud computing, peer-to-peer computing, and cluster 
computing within a simple framework.   Computers running Seattle are protected 
from malicious and misbehaving code, making it safe to contribute
resources from multi-use computers.
Seattle has resources available for students to use on about a thousand 
computers worldwide.

We are currently working to extend Seattle to improve Seattle's NAT 
traversal and to provide better aggregate restrictions on Seattle traffic.


%ACKNOWLEDGMENTS are optional
%\section{Acknowledgments}


%
% The following two commands are all you need in the
% initial runs of your .tex file to
% produce the bibliography for the citations in your paper.

{%\small
\bibliographystyle{abbrv}
\bibliography{bibdata}}  % paper.bib is the name of the Bibliography in this case

%% THIS IS SIGPROC-SP.TEX - VERSION 2.9
% WORKS WITH V3.0SP OF ACM_PROC_ARTICLE-SP.CLS
% MARCH 2007
%
% It is an example file showing how to use the 'acm_proc_article-sp.cls' V3.0SP
% LaTeX2e document class file for Conference Proceedings submissions.
% ----------------------------------------------------------------------------------------------------------------
% This .tex file (and associated .cls V3.0SP) *DOES NOT* produce:
%       1) The Permission Statement
%       2) The Conference (location) Info information
%       3) The Copyright Line with ACM data
%       4) Page numbering
% ---------------------------------------------------------------------------------------------------------------
% It is an example which *does* use the .bib file (from which the .bbl file
% is produced).
% REMEMBER HOWEVER: After having produced the .bbl file,
% and prior to final submission,
% you need to 'insert'  your .bbl file into your source .tex file so as to provide
% ONE 'self-contained' source file.
%
% Questions regarding SIGS should be sent to
% Adrienne Griscti ---> griscti@acm.org
%
% Questions/suggestions regarding the guidelines, .tex and .cls files, etc. to
% Gerald Murray ---> murray@acm.org
%
% For tracking purposes - this is V2.9SP - MARCH 2007

\documentclass{sig-alternate}
\usepackage{verbatim}
\usepackage{url}

%\pagenumbering{arabic}

%-- Begin patch area for accents in 'Author Block' area - may be needed by some authors / but not all
\DeclareFixedFont{\auacc}{OT1}{phv}{m}{n}{12}   % Needed for "Author Block" accents - Patch by Gerry 3/21/07
\DeclareFixedFont{\afacc}{OT1}{phv}{m}{n}{10}   % Needed for "Author Block" accents in the affiliation/address line - Patch by Gerry 3/21/07
%--
\newtheorem{remark}{Remark}

% I'm using this so I can easily mark parts that are moved to the TR...
\newcommand{\movetotr}[1]{{}}

\begin{document}
\conferenceinfo{SIGCSE'09,} {March 3--7, 2009, Chattanooga, Tennessee, USA.} 

\CopyrightYear{2009}

\crdata{978-1-60558-183-5/09/03} 

\title{Seattle: A Platform for Educational Cloud Computing\thanks{This work 
was partially supported by NSF Grant CNS-0834243}
}
%
% You need the command \numberofauthors to handle the 'placement
% and alignment' of the authors beneath the title.
%
% For aesthetic reasons, we recommend 'three authors at a time'
% i.e. three 'name/affiliation blocks' be placed beneath the title.
%
% NOTE: You are NOT restricted in how many 'rows' of
% "name/affiliations" may appear. We just ask that you restrict
% the number of 'columns' to three.
%
% Because of the available 'opening page real-estate'
% we ask you to refrain from putting more than six authors
% (two rows with three columns) beneath the article title.
% More than six makes the first-page appear very cluttered indeed.
%
% Use the \alignauthor commands to handle the names
% and affiliations for an 'aesthetic maximum' of six authors.
% Add names, affiliations, addresses for
% the seventh etc. author(s) as the argument for the
% \additionalauthors command.
% These 'additional authors' will be output/set for you
% without further effort on your part as the last section in
% the body of your article BEFORE References or any Appendices.

%\numberofauthors{5} %  in this sample file, there are a *total*

% of EIGHT authors. SIX appear on the 'first-page' (for formatting
% reasons) and the remaining two appear in the \additionalauthors section.
%
\author{
%Author Names Removed for Anonymous Submission
Justin Cappos~~~~Ivan Beschastnikh~~~~Arvind Krishnamurthy~~~~Tom Anderson\\[1ex]
       \affaddr{Department of Computer Science and Engineering, University of Washington}\\
       \affaddr{Seattle, WA 98105, U.S.A.}\\
       \email{\{justinc, ivan, arvind, tom\}@cs.washington.edu}
}
\date{}
%\hyphenation{Com-put-er-Comm-uni-ca-tion Comm-uni-ca-tion}

\maketitle
\abstract

%As many organizations outsource hardware and maintainence and instead
%focus on software, % IB: do we need give a reason for the transition?

Cloud computing is rapidly increasing in popularity. Companies such as
RedHat, Microsoft, Amazon, Google, and IBM are increasingly funding
cloud computing infrastructure and research, making it important for
students to gain the necessary skills to work with cloud-based
resources. This paper presents a free, educational research platform
called Seattle that is community-driven, a common denominator for
diverse platform types, and is broadly deployed.

Seattle is community-driven --- universities donate available compute
resources on multi-user machines to the platform. These donations can
come from systems with a wide variety of operating systems and
architectures, removing the need for a dedicated infrastructure.
% to support Seattle.

Seattle is also surprisingly flexible and supports a variety of
pedagogical uses because as a platform it represents a common denominator 
for cloud computing, grid computing, peer-to-peer networking,
distributed systems, and networking. Seattle programs are portable.
Students' code can run across different operating systems and
architectures without change, while the Seattle programming language
is expressive enough for experimentation at a fine-grained level. Our
current deployment of Seattle consists of about one thousand computers
that are distributed around the world. We invite the computer science
education community to employ Seattle in their courses.

% mention planetlab

% arvind's comments:
%Community-driven (resources are provided by the community).
%
%Common denominator for wide range of computing types
%
%Free
%
%Broad deployment (many hundreds of systems)

%Cloud computing is rapidly increasing in popularity as many organizations
%outsource hardware and maintainence and instead focus on software.   With
%major industry players like RedHat, Microsoft, Amazon, Google, and IBM
%investing heavily in cloud computing infrastructure and research, it is 
%important that students gain the skills they need to work with cloud-based 
%resources.   
%We have built a cloud computing platform called Seattle for the
%purpose of allowing students to experiment with cloud computing in a safe,
%portable, and cost-free environment.   Computers that run Seattle are protected
%from buggy and malicious programs by a sandboxing mechanism that provides 
%security and performance isolation.  Students that use Seattle can write 
%programs that use resources on around one thousand computers running at 
%hundreds of universities distributed around the world.   An instructor can
%use Seattle to demonstrate the difference between cluster computing 
%(possibly remote computers with low inter-computer latency), cloud 
%computing (collaborative computers where at least some systems have low
%latency to the source), peer-to-peer computing (randomly distributed 
%computers), and other in-between scenarios.

% A category with the (minimum) three required fields
\category{K.3.2}{Computer and Information Science Education}{Computer science education}
\category{C.2.4}{Computer-Communi-cation Networks}{Distributed Systems}[client/server, distributed applications]
\category{C.4}{Computer Systems Organization}{Performance of Systems}[design studies, measurement techniques]

%A category including the fourth, optional field follows...\category{D.2.8}{Software Engineering}{Metrics}[complexity measures, performance measures]
\terms{Experimentation, Measurement, Performance, Security}

\keywords{Cloud computing, peer-to-peer computing, cluster computing, distributed computing}
\section{Introduction}

Cloud computing is rapidly increasing in popularity as many organizations
outsource hardware and maintenance and instead focus on 
software~\cite{AmazonEC2, AmazonS3, GoogleAppEngine, GoogleDocs, HPCloud, 
IBMBlueCloud, MicrosoftMemo}.   
%Amazon
%has built an infrastructure for Elastic Cloud Computing (EC2)~\cite{AmazonEC2} 
%on top of RedHat servers.   Google has provided a programming API and platform
%called AppEngine~\cite{GoogleAppEngine} that transparently provides scalability
%while hiding locality information from applications.   Google also provides
%applications like Google Docs~\cite{GoogleDocs} which are built using cloud
%computing principles.  Other major industry players like 
%Microsoft~\cite{MicrosoftMemo} and IBM~\cite{IBMBlueCloud} have announced their
%own cloud computing platforms and HP, Intel, and Yahoo!~\cite{HPCloud} have 
%announced a joint pushes for cloud computing research.   
%%%%%
%%%% An attempt where I talk about client software as well as cloud differences
%However, despite all of the focus on cloud computing, there
%is a lot of disparity in industry in terms of what cloud computing 
%really means.   RedHat / Amazon's EC2~\cite{AmazonEC2} provides cloud
%computing as a collection of Linux boxes along with
%storage functionality~\cite{AmazonS3} while leaving the client interaction
%with the cloud up to the developer.   Google provides automatic scaling of
%cloud computing programs written to a custom programming 
%API~\cite{GoogleAppEngine} and tends to have clients access cloud resources 
%using their web browser~\cite{GoogleDocs}.   Microsoft views it as a 
%virtualization layer between the hardware and OS and is releasing 
%developer toolkits to be used in a ``software plus service'' architecture where
%the client runs custom cloud software on their local 
%machine~\cite{MicrosoftMemo}.
%%%%%
However, despite the attention, there is a lot of disparity in what
cloud computing means. RedHat / Amazon's EC2~\cite{AmazonEC2} provides
cloud computing as a collection of Linux machines with storage
functionality~\cite{AmazonS3}. Google's platform for cloud computing
hides locality and scalability issues from the programmer who writes
programs to a custom programming API~\cite{GoogleAppEngine}. Microsoft
views it as a virtualization layer between the hardware and the OS and
is releasing a developer toolkit for providing the user with
``software plus service.''~\cite{MicrosoftMemo}

We provide an educational platform called Seattle that is a common
denominator for a wide range of these definitions. Seattle's simple to
learn programming language, a subset of the Python language, is
expressive enough to allow students to build algorithms for
inter-machine interaction (like a global store or a DHT). As a result,
Seattle is useful in many pedagogical contexts ranging from courses in
cloud computing, networking, and distributed systems, to parallel
programming, grid computing, and peer-to-peer computing.

Seattle is a community-driven effort that depends on resources donated
by users of the software (and as such is free to use). A user can
install Seattle onto their personal computer to enable Seattle
programs to run using a portion of the computer's resources. Seattle
programs are sandboxed and securely isolated from other programs
running on the same computer. Seattle provides hard resource
guarantees that an erroneous or malicious program cannot circumvent.

In addition, Seattle runs on a variety of different operating systems
and architectures including Windows, Mac OS-X, Linux, FreeBSD, and even
portable devices like Nokia N800s and jail broken iPhones. Code written
for Seattle is automatically (and transparently) portable to different
architectures and runs the same across all systems.

Seattle has a preexisting base of installed computers and is already
widely deployed on almost one thousand computers that are spread
across hundreds of universities worldwide. Seattle users can run their
programs on computers spanning the Internet -- a feature that is
currently being used by several classes at major universities.

%While Seattle is useful as a cloud computing platform, there are
%other important pedagogical concepts that Seattle can illustrate.
%The first concept is that the characteristics of links between computing 
%resources 
%matters.  In other words, two computers in a data center behave differently 
%than two computers connected by a trans-atlantic cable.   The second concept is
%that the grouping of resources is important.   It is possible to efficiently 
%solve different sets of problems with 100\% of one CPU than 10\% of 10 
%computers.   The third concept is that failure handling is important.   In 
%globally distributed applications, communications failures due to routing
%black holes or DNS issues are commonplace~\cite{Katz_NSDI_2008}.   Applications
%that seek to provide high uptime must handle failures well.

%\subsection{Map}
This paper describes the architecture of the Seattle cloud computing
platform (Section~\ref{sec-architecture}) including the
programming API (Section~\ref{sec-API}), 
the sandboxing mechanism (Section~\ref{sec-repy}), 
the control of sandboxes on a
host computer (Section~\ref{sec-nodemanager}), and the way in which
students manage their running programs
(Section~\ref{sec-experimentmanager}). Following this, we describe the
computational resources available to classes using Seattle and how we
expect this platform to grow in the future
(Section~\ref{sec-deployment}).  Next, we provide some example
assignments to demonstrate how Seattle can be used in courses
(Section~\ref{sec-assignments}). We then discuss related work
(Section~\ref{sec-relatedwork}) and conclude
(Section~\ref{sec-conclusion}).




\section{Architecture}
\label{sec-architecture}

To use Seattle, the instructor creates an account on our website and obtains
an installer.   The machines that run the installer (such as computers in 
the universities computer lab) donate resources that are credited to the 
instructor.   The instructor can then obtain resources on machines around the 
world.   As of December 1st, 2008 the current policy is that for each donation,
the instructor receives resources on ten other computers.   However, the 
instructor can delegate those resources either directly to students or
to TAs who do more fine-grained delegation.   Students and TAs download
a toolkit and then experiment with their resources.

Seattle's architecture is comprised of several components. At the
lowest level the \emph{sandbox} component guarantees security and
resource control for an individual program. Programs are written to
the Seattle API in a subset of the Python programming language. This
API provides portable access to low level operations (like opening
files and sending messages). At a higher layer, the \emph{node
  manager} determines which sandboxed programs get to run on the local
computer. A public key infrastructure is used to authorize control
over sandboxed programs. Lastly, the \emph{experiment manager} lets
students control their program instances across computers.

\subsection{Seattle API}
\label{sec-API}

Seattle provides a programming API for low-level operations (like
writing to files or sending network messages) and maintains program
portability using an abstraction layer. Platform specific code below
this abstraction layer handles non-portable operations enabling
unmodified programs to run on a wide variety of
platforms. % This also allows the majority of the sandbox to be reused
%across different architectures.


The API consists of five categories: file, network, timer, locking,
and miscellaneous. The file API calls enable limited access to the
local computer's persistent storage (interacting only with files in a 
single directory). 
%the calls to
%create, delete, open, read and write sandboxed files, as well as calls
%to list directory contents in the sandbox. 
The network API calls
provide the local IP address, perform a DNS lookup, enable sending and
receiving of UDP messages, and managing of and communicating over TCP
connections. The timer API calls enable the programmer to put the
current thread to sleep and to schedule functions to be called at
later times. For example, the programmer can register an event to
periodically send a heartbeat message to another computer. The locking
functions allow the programmer to handle concurrency in their program
(as common state may be accessed and modified by multiple threads at
the same time). The miscellaneous API calls allow the programmer to
exit the program, to generate random numbers, and to provide the
amount of time the program has been running.

% initiation/termination of and listening for TCP connections

\subsection{The Sandbox}
\label{sec-repy}

The sandbox's primary goal is to securely execute user code. There are
two aspects to this --- preventing insecure actions and limiting
resource consumption. To prevent insecure actions the sandbox hooks into 
the Python parser and reads the program's parse tree. Only actions that
the sandbox can verify as safe may execute.

To control resource consumption on the host the sandbox interposes on
all API calls made by a program. The sandbox monitors the overall use
of resources like CPU, memory, and disk space to ensure the program
does not exceed its bounds. Each API call that uses a monitored
resource is evaluated before being granted or denied the resource. The
sandbox also restricts the rate at which API calls are performed.

The restrictions and resource limits of the sandbox are configurable and
may restrict different programs in different ways. For example, one 
program may only be
allowed to receive UDP packets on port 11111, while another program
may be restricted to receiving UDP packets on port 22222. This enables
multiple sandboxes on the same computer to host programs controlled by
different users.

%We provide an interface where the program can execute unsafe operations (such 
%as open files, send network traffic, etc.).   Our sandbox verifies that the 
%program is performing a safe operation (like opening a file in the program's 
%sandbox) instead of a malicious action (like opening the user's credit card 
%information).   This is done by fine grained checks on the arguments passed 
%to individual calls.   Once again we err on the side of caution (for example, 
%restricting the names of files the program can open) to prevent the program 
%from escaping the sandbox.
%
%However, not every safe action should be allowed by every program.   It may 
%be that two programs would like to run on the same system and each program 
%has been allocated its own port.   The programs should not be able to use 
%the other program's port or other ports on the system.   To prevent this 
%repy has per-program fine-grained restrictions that control the use of 
%calls.   Program A can be restricted to port 12345 while Program B is 
%restricted to port 54321.   Similarly, a Program C that does not need 
%to write files to the disk can be prevented from using file operations all 
%together!   This meets our design goal of a multi-user programming environment
%since students are isolated from each other.
%
%
%\subsubsection{Resource Limits}
%\label{sec-resourcelimits}
%
%Another important type of isolation is resource isolation.   One program that 
%runs on a computer should not be able to consume enough resources to 
%impact the execution of other programs or the host computer.   To this end, 
%repy limits resource consumption through several mechanisms.   For 
%resources that renew themselves over time (like CPU, network send rate, file 
%write rate, etc.) the program is paused if it tries to over use the resource 
%so that the performance of the system does not suffer.   After a suitable 
%amount of time, the program is restarted.    If the resource is not renewable 
%(like the number of open files, memory use, disk used, etc.) then either the 
%function raises an exception or the program is killed.   This meets our
%design goal of preventing the 
%program from negatively impacting the performance of other programs running 
%on the same system.
%


\subsection{Node Manager}
\label{sec-nodemanager}
While useful in itself, the sandbox is part of a larger ecosystem.
The sandbox isolates a specific running program on a host computer,
but does not address how that program is started, which programs are
run, and who has permission to run a program. Such functionality is
provided by the node manager, which manages sandboxed running programs
as part of what we call \emph{vessels}. The node manager stores
information about the vessels it controls and allows vessels to be
started, stopped, combined, split, and changed.


\subsubsection{Vessels}

A vessel is a controlled environment for running code (implemented
using the Seattle sandbox).  Intuitively, a vessel includes the
program's sandbox and the node manager state (such as the resources
and restrictions assigned to the program).  Vessels have well defined
boundaries that prevent them from interfering with one another (for
example, different vessels have their own disjoint set of network
ports).  Each vessel has associated with it a restrictions file, a
stop file, and a log.  The restrictions file lists what the vessel can
and cannot do (such as the network ports that can be used) along with
the amount of each resource the vessel may use.  The stop file enables
the node manager to stop the vessel (by creating a file with that
name). The log is a circular buffer written by the vessel to
communicate useful information to the vessel owner. The log helps the
programmer to diagnose failures and to capture program state for
off-line analysis.

% (who has the initial set of resources since they installed
% the software)
A common scenario is for a student to obtain a vessel from their
instructor. The student then decides the program they want to run in
their vessel. To do this, she directs the experiment manager to
install the program on a node. The experiment manager uploads the
program to the student's vessel (along with any data files) and
executes the program in the vessel. The student also can easily perform
this action on groups of vessels spread across many different nodes.
The student can then monitor the status of her program by looking at a 
status indicator provided by querying the node manager (coarse-grained 
monitoring) or by downloading information 
about the program from its circular log buffer (fine-grained monitoring).  
The user can also stop the vessel while retaining all of the state so that 
she can examine data files and logs.   

In a more complex example an instructor splits a single vessel on a
node into multiple vessels, and
assigns each vessel to a student
in the class. Vessels may also be combined for flexibility. For example,
the students may be allowed to work in groups. Once the groups are
formed, some of the students may decide to combine their vessels so as
to get more resources in a single vessel.

% ib: do we need to describe offcut resources? seems too detailed.
% jac: commented out
%
%When a vessel is split, the quantity of resources isn't exactly
%divided between the two new vessels because of inherent overhead
%inherent in managing a vessel. To capture this effect, there is a
%resource specification for offcut resources.  Whenever a vessel is
%split, an amount of resources equal to the offcut resources is lost.
%Whenever two vessels are joined, the vessels gain resources equal to
%the offcut resources.

\subsection{Locating Seattle Nodes}
\label{sec-locationservice}

It is important to note that there is nothing parallel or distributed
about the node manager. The node manager only manages the vessels on
the local system. To facilitate global location of resources, the node
manager inserts a key/value pair into two different public DHTs
(OpenDHT~\cite{Rhea_SIGCOMM_2005}) every five minutes. The inserted key is
the owner's public key and the value is the local computer's IP
address. This allows a user to lookup their public key to find
the nodes with vessels they control without needing to keep track of
these nodes on their own.

\subsection{Experiment Manager}
\label{sec-experimentmanager}

The experiment manager is the main tool in the toolkit that students use
to interact with Seattle. The experiment manager
transparently handles discovery of vessels the user controls by
querying the DHT, and communicates with remote node managers to perform
actions on the user's behalf.

The experiment manager provides the user with a shell interface
(similar to PLuSH~\cite{Albrecht_LISA_2007}) in which the user can
issue commands. For example, users can install software in their
vessels -- the experiment manager uploads a program into the vessels
the user specifies. Users can also start and stop vessels, or report
on the status of a vessel. A vessel's status can be fresh (has never
run a program), started (is running a program), stopped (was requested
to stop), or terminated (terminated due to a normal exit or an unhandled 
program exception). When a vessel has failed
(perhaps due to a bug in the student's code), exception information
with a stack trace of the fault is logged. The student can use
the experiment manager to find the failed program instances to inspect
their logs or to see an exception's stack trace.



\section{Deployment}
\label{sec-deployment}

Since it is unclear what the future of cloud computing will be, we are
interested in providing students with the most diverse set of
resources possible. Seattle runs on Linux, FreeBSD, Mac OS-X, XO (one
laptop per child)\footnote[1]{Some threading libraries and operating
  systems do not provide accurate CPU and/or memory information. As a
  result, certain resources cannot be effectively sandboxed}, and
Windows platforms. Seattle also runs on mobile devices like the
iPhone\footnotemark[1] (if jail broken) and Nokia N800\footnotemark[1].
We are interested in adding support for other platforms as users
express interest in running Seattle on new platforms.

In addition to platform diversity, network connection diversity is
also important. Seattle is already widely deployed at universities
around the world. An instance of Seattle is running on each 
PlanetLab~\cite{PlanetLab} node, giving a presence on close to one thousand
nodes at hundreds of universities.  

PlanetLab provides access to computers at a large number of well
connected locations, although most only have two computers per site
available. As we are now publicly releasing Seattle for educational
use, we expect to see an increase in resource diversity.  We
anticipate that universities will install Seattle on most of the
computers in their lab. Such deployments will provide a distributed
environment that effectively emulates cluster computing. We also
anticipate that many students will load Seattle on their home
computers. This will allow for emulation of peer-to-peer computing as
home computers will typically have connectivity characteristics
representative of the average Internet user.

We believe that the diversity of platforms and network connections
enables a wide range of pedagogical uses. Students can experiment with
cluster computing, grid computing, peer-to-peer, and cloud computing
by simply varying where their program is deployed. Additionally, the
same program will run in any of these environments. Of course, the
characteristics of the environment will determine the efficiency and
scalability of the employed distributed algorithm --- a crucial
distributed systems lesson for students.


\section{Seattle in the Classroom}
\label{sec-assignments}

This section describes educational resources available to the student and 
instructor.   The resources are divided into two areas.   There is a student
portal which contains resources for students who are learning the basics of
Seattle.   There is also an educator portal that contains resources for 
educators to help them use Seattle in the classroom.   

\subsection{Student Resources}
To aid students learning Seattle, we provide a student portal with
tutorials describing how to program and use Seattle, documentation for the API,
a resources and restrictions guide, and other documentation.   The tutorials 
provide code snippets demonstrating the API, and explains how to use the 
experiment manager to perform different tasks.
Our experience has shown that students can quickly learn the Python
programming language~\cite{611980, Cappos_WORLDS_2005} and quickly learn
to program in Seattle.   Our experience is that undergraduates who had no 
previous networking experience can implement programs like overlay 
multicast and TCP forwarding with a few hours of effort after completing 
the tutorial.
% JAC: Should this be removed? Is mentioning ease of learning a distraction?
% ib: i think this should stay, an important characteristic for teaching with Seattle

To illustrate how easy it is to program Seattle, 
A popular first project for networking students is the Echo
client/server. The two \emph{complete} Seattle programs are both
concise and simple. The echo client is just 6 lines of code:

{\scriptsize
\begin{verbatim}
# Handle an incoming message
def got_reply(srcip, srcport, mess, ch):
  print 'received:',mess,"from",srcip,srcport

if callfunc == 'initialize':
  # when a message arrives on my IP, port 43210, 
  # start an event to call the function 'got_reply'
  recvmess(getmyip(), 43210, got_reply)  
  # send a hello message to my IP, port 54321
  sendmess(getmyip(), 54321, 'hello', getmyip(), 43210)  
  # exit in one second
  settimer(1,exitall,())
\end{verbatim}
}

The echo server consists of just 4 lines of code:

{\scriptsize
\begin{verbatim}
# Handle an incoming message
def got_message(srcip, srcport, mess, ch):
  sendmess(srcip,srcport,mess)

if callfunc == 'initialize':
  recvmess(getmyip(), 54321, got_message)  
\end{verbatim}
}


Furthermore, a first project for many students in distributed systems is to 
measure the connectivity characteristics between pairs of
computers~\cite{AllPairsPing}.  The Appendix lists a complete 30 line Seattle
program that performs an all-pairs-ping and displays its results in a webpage 
when contacted.


\subsection{Educator Resources}

To aid instructors in integrating Seattle into their existing curriculum, 
we also provide an educator portal.   The first item on the educator 
portal is a description of instructor experiences with Seattle.   After
Seattle has been used, we have asked the students to fill out a survey 
outlining how they felt the platform impacted their experience.   We recorded
the results on the web page.

Besides information about experiences with Seattle, the educator portal also 
contains course materials such as handouts and example assignments.   
One special-purpose assignment called the TakeHomeAssignment requires 
no programming and takes about 1 hour to do.   The purpose of the 
TakeHomeAssignment is to show the student or instructor the practical effects 
of non-transitive connectivity and NATs on the Internet today, while 
introducing them to Seattle.

In addition, the educator portal provides a collection of ready-to-go 
programming assignments for use with Seattle.   The assignments include implementing a 
reliable protocol on top of UDP, performing overlay routing using link-state 
routing on the Internet, building application level services like a webserver, and 
understand layering of services by creating a chat server that operates 
over HTTP.   

The educator portal also provides assignment ideas that are appropriate 
for course long projects or graduate assignments.   For example, students 
may implement a DHT (such as Chord~\cite{Stoica_SIGCOMM_2001}) to better 
understand non-transitive connectivity.   A simple implementation will work
well on LAN environments but will fail horribly on the Internet due to 
non-transitive connectivity and other network effects.   After measuring
these effects and then understanding the reason behind Chord's poor 
performance, the students can discuss solutions to these
problems. Students can then implement and test these solutions to achieve
better performance and reliability. 
This assignment emphasizes good software engineering practices (since code is
reused), that test and deployment environments may differ significantly, 
and encourages students to come up with unique solutions to the problem yet 
is easily evaluated using a small set of metrics.



\section{Related Work}
\label{sec-relatedwork}

%As the
%future of cloud computing is far from clear, Seattle benefits courses
%varying in their coverage of parallel processing and distributed
%computing~\cite{SIGCSE98}.

%Seattle's Python-based interface is simple to learn, and easy to
%master. Yet, it retains much power in the hands of the programmer.
%Seattle can provide a fun experience for students without them having
%to master advanced networking API features nor organize the
%distribution of their code to other platforms or having to organize a
%testbed of machines.

There are many different cloud computing platforms in use today.
Amazon runs a cloud of RedHat servers to provide computing
resources~\cite{AmazonEC2},
which %and a storage back end~\cite{AmazonS3}.
are similar in purpose to Seattle but provide a virtual machine
instead of a programming language instance. This leads to better
performance, but is not flexible to support donated resources and is
not free. Their storage back end is functionally similar to the global
data store proposed as an assignment on Seattle's educator portal.

Microsoft proposes a software plus services~\cite{MicrosoftMemo}
architecture where the cloud is used as an auxiliary to enhance the
capabilities of local software. While Microsoft has announced the
pending availability of a developer toolkit, to the best of our
knowledge it is not available at this time. % for further comparison.

Google provides a cloud computing-like infrastructure with
AppEngine~\cite{GoogleAppEngine}, which executes programs written in a
constrained version of the Python language and supports high level
abstractions (like a global database). While useful for building
locality-oblivious web applications, its transparent handling of
scalability and locality makes it unsuitable for teaching these
fundamental distributed systems topics.

IBM has announced plans for a cloud computing product called ``Blue
Cloud''~\cite{IBMBlueCloud}, which supports OS images using Xen and
PowerVM Linux. Blue Cloud also supports Hadoop~\cite{Hadoop} for
MapReduce-type query processing and is intended to support high
performance computing. %so that organizations can migrate supercomputer
%tasks to Blue Cloud.
Hadoop has also been used to teach cluster computing for large data
processing~\cite{SIGCSE08}. The MapReduce~\cite{MapReduce} paradigm
used by Hadoop simplifies distributed data processing. However, this
simplifying abstraction also limits the scope of systems concepts that
may be taught with Hadoop. We believe that a more general platform
should be used to teach the system concepts that power Hadoop's
implementation. These concepts may then be applied more broadly to
other distributed computing abstractions, and cloud computing more
generally.

In addition to cloud computing, there are a variety of grid computing
and volunteer computing platforms. Globus~\cite{Globus} is a popular
Grid toolkit, which has been used to build a variety of
service-oriented applications. BOINC~\cite{BOINC} is a volunteer
computing platform supporting the SETI@Home and Folding@Home projects.
BOINC leverages donated CPU cycles for computation, particularly spare
resources on home machines. %To BOINC, getting 1\% of the
%time on 1000 computers is similar to getting 100\% on 10 computers
%(for similar computers). With Seattle, it is much more useful to get
%1\% on 1000 computers because the research and educational programs
%are trying to understand how to write scalable programs that run on
%many computers at the same time.
Globus and BOINC both target distributed computation and strive to
hide locality and similar information from the programmer. Seattle is
lighter-weight software that exposes locality and is therefore suited
for students in distributed systems courses. Seattle also comes with a
widely accessible ready-to-use platform of thousands of machines. We
are not aware of any Globus- or BOINC-powered testbeds available for
educational use.

%The problems and
%programs become interesting when there are thousands of computers.

%% Also most volunteer computing platforms like BOINC use CPU cycles only
%% when the system would be idle so as to avoid interfering with the
%% user.  BOINC typically uses almost all of the CPU when it's idle.
%% With Seattle, it is more desirable to constantly have a small
%% percentage of the resources (to allow a more constant set of resources
%% to be used over a longer time period).

Limited compute resources have been a key constraint in teaching
distributed systems~\cite{SIGCSE94}. Seattle is architected to do this
safely and efficiently. % across a variety of architectures.
Recent efforts engaged undergraduates in distributed computing with
simple-to-use platforms designed for cluster
computing. %~\cite{SIGCSE07}.
The DCEZ platform offers a simple interface that students can use
without any prior knowledge of distributed
computing~\cite{SIGCSE07}. %With minimal assistance, students were
%shown to install and use a small cluster with DCEZ within 5 to 10
%minutes.
We have likewise endeavored to make Seattle simple to use, but
target teaching of distributed systems issues that arise at Internet
scales. Lastly, because Seattle runs on a variety of embedded
platforms with limited resources, such as cellular phones and PDAs,
our platform complements prior work on platforms for teaching
ubiquitous computing~\cite{SIGCSE03}.


\section{Conclusion and Future Work}
\label{sec-conclusion}
This work presents the educational networking platform Seattle. Seattle is a
free, portable, and lightweight platform using donated computational
resources. Seattle enables students to learn the concepts of networking and
distributed systems on computers spread around the Internet.   Seattle
can also emulate cloud computing, peer-to-peer computing, and cluster 
computing within a simple framework.   Computers running Seattle are protected 
from malicious and misbehaving code, making it safe to contribute
resources from multi-use computers.
Seattle has resources available for students to use on about a thousand 
computers worldwide.

We are currently working to extend Seattle to improve Seattle's NAT 
traversal and to provide better aggregate restrictions on Seattle traffic.


%ACKNOWLEDGMENTS are optional
%\section{Acknowledgments}


%
% The following two commands are all you need in the
% initial runs of your .tex file to
% produce the bibliography for the citations in your paper.

{%\small
\bibliographystyle{abbrv}
\bibliography{bibdata}}  % paper.bib is the name of the Bibliography in this case

%% THIS IS SIGPROC-SP.TEX - VERSION 2.9
% WORKS WITH V3.0SP OF ACM_PROC_ARTICLE-SP.CLS
% MARCH 2007
%
% It is an example file showing how to use the 'acm_proc_article-sp.cls' V3.0SP
% LaTeX2e document class file for Conference Proceedings submissions.
% ----------------------------------------------------------------------------------------------------------------
% This .tex file (and associated .cls V3.0SP) *DOES NOT* produce:
%       1) The Permission Statement
%       2) The Conference (location) Info information
%       3) The Copyright Line with ACM data
%       4) Page numbering
% ---------------------------------------------------------------------------------------------------------------
% It is an example which *does* use the .bib file (from which the .bbl file
% is produced).
% REMEMBER HOWEVER: After having produced the .bbl file,
% and prior to final submission,
% you need to 'insert'  your .bbl file into your source .tex file so as to provide
% ONE 'self-contained' source file.
%
% Questions regarding SIGS should be sent to
% Adrienne Griscti ---> griscti@acm.org
%
% Questions/suggestions regarding the guidelines, .tex and .cls files, etc. to
% Gerald Murray ---> murray@acm.org
%
% For tracking purposes - this is V2.9SP - MARCH 2007

\documentclass{sig-alternate}
\usepackage{verbatim}
\usepackage{url}

%\pagenumbering{arabic}

%-- Begin patch area for accents in 'Author Block' area - may be needed by some authors / but not all
\DeclareFixedFont{\auacc}{OT1}{phv}{m}{n}{12}   % Needed for "Author Block" accents - Patch by Gerry 3/21/07
\DeclareFixedFont{\afacc}{OT1}{phv}{m}{n}{10}   % Needed for "Author Block" accents in the affiliation/address line - Patch by Gerry 3/21/07
%--
\newtheorem{remark}{Remark}

% I'm using this so I can easily mark parts that are moved to the TR...
\newcommand{\movetotr}[1]{{}}

\begin{document}
\conferenceinfo{SIGCSE'09,} {March 3--7, 2009, Chattanooga, Tennessee, USA.} 

\CopyrightYear{2009}

\crdata{978-1-60558-183-5/09/03} 

\title{Seattle: A Platform for Educational Cloud Computing\thanks{This work 
was partially supported by NSF Grant CNS-0834243}
}
%
% You need the command \numberofauthors to handle the 'placement
% and alignment' of the authors beneath the title.
%
% For aesthetic reasons, we recommend 'three authors at a time'
% i.e. three 'name/affiliation blocks' be placed beneath the title.
%
% NOTE: You are NOT restricted in how many 'rows' of
% "name/affiliations" may appear. We just ask that you restrict
% the number of 'columns' to three.
%
% Because of the available 'opening page real-estate'
% we ask you to refrain from putting more than six authors
% (two rows with three columns) beneath the article title.
% More than six makes the first-page appear very cluttered indeed.
%
% Use the \alignauthor commands to handle the names
% and affiliations for an 'aesthetic maximum' of six authors.
% Add names, affiliations, addresses for
% the seventh etc. author(s) as the argument for the
% \additionalauthors command.
% These 'additional authors' will be output/set for you
% without further effort on your part as the last section in
% the body of your article BEFORE References or any Appendices.

%\numberofauthors{5} %  in this sample file, there are a *total*

% of EIGHT authors. SIX appear on the 'first-page' (for formatting
% reasons) and the remaining two appear in the \additionalauthors section.
%
\author{
%Author Names Removed for Anonymous Submission
Justin Cappos~~~~Ivan Beschastnikh~~~~Arvind Krishnamurthy~~~~Tom Anderson\\[1ex]
       \affaddr{Department of Computer Science and Engineering, University of Washington}\\
       \affaddr{Seattle, WA 98105, U.S.A.}\\
       \email{\{justinc, ivan, arvind, tom\}@cs.washington.edu}
}
\date{}
%\hyphenation{Com-put-er-Comm-uni-ca-tion Comm-uni-ca-tion}

\maketitle
\input{abstract}
% A category with the (minimum) three required fields
\category{K.3.2}{Computer and Information Science Education}{Computer science education}
\category{C.2.4}{Computer-Communi-cation Networks}{Distributed Systems}[client/server, distributed applications]
\category{C.4}{Computer Systems Organization}{Performance of Systems}[design studies, measurement techniques]

%A category including the fourth, optional field follows...\category{D.2.8}{Software Engineering}{Metrics}[complexity measures, performance measures]
\terms{Experimentation, Measurement, Performance, Security}

\keywords{Cloud computing, peer-to-peer computing, cluster computing, distributed computing}
\input{introduction}
\input{architecture}
\input{deployment}
\input{assignments}
\input{related}
\input{conclusion}

%ACKNOWLEDGMENTS are optional
%\section{Acknowledgments}


%
% The following two commands are all you need in the
% initial runs of your .tex file to
% produce the bibliography for the citations in your paper.

{%\small
\bibliographystyle{abbrv}
\bibliography{bibdata}}  % paper.bib is the name of the Bibliography in this case

%\input{paper.bbl}


\input{appendix}

\balancecolumns
\end{document}



\appendix

The following Seattle program measures network latency to a list of IP
addresses and displays a webpage showing the latency to each
node. This program has 30 lines of code, with 11 lines of
comments. Without the webpage functionality the program is only 21
lines of code. This is a complete Seattle program --- no code is
omitted or abbreviated.

%\caption{This Seattle program pings a list of nodes and gathers latency 
%information.   The user can browse the computer to see a HTML document
%that displays the latency information in a table. }

%\begin{figure}
{
%  # I'm going to write a HTML header first...
%  # and we're done, so let's close this connection...
%  # now the footer...
\scriptsize
\begin{verbatim}
# send a probe message to each neighbor
def probe_neighbors(port):
  for neighborip in mycontext["neighborlist"]:
    mycontext['sendtime'][neighborip] = getruntime()
    sendmess(neighborip, port, 'ping',getmyip(),port)
  # probe again in 10 seconds
  settimer(10,probe_neighbors,(port,))

# handle an incoming message
def got_message(srcip,srcport,mess,ch):
  if mess == 'ping':
    sendmess(srcip,srcport,'pong')
  elif mess == 'pong':
    # elapsed time is now - time when I sent the ping
    mycontext['latency'][srcip] = getruntime() - \
                         mycontext['sendtime'][srcip]

# display a web page with the latency information
def show_status(srcip,srcport,connobj, ch, mainch): 
  connobj.send("<html><head><title>Latency Information</title>"+ \
           "</head><body><h1>Latency information from "+ \
           getmyip()+' </h1><table border="1">')
  # list a row for each node we are talking to 
  for neighborip in mycontext['neighborlist']:
    if neighborip in mycontext['latency']:
      connobj.send("<tr><td>"+neighborip+"</td><td>"+ \
           str(mycontext['latency'][neighborip])+"<td></tr>")
    else:
      connobj.send("<tr><td>"+neighborip+"</td><td>Unknown<td></tr>")
  connobj.send("</table></html>")
  connobj.close()

if callfunc == 'initialize':
  # this holds the response information (i.e. when nodes responded)
  mycontext['latency'] = {}
  # this remembers when we sent a probe
  mycontext['sendtime'] = {}
  # get the nodes to probe
  mycontext['neighborlist'] = []
  for line in file("neighboriplist.txt"):
    mycontext['neighborlist'].append(line.strip())
  # call gotmessage whenever receiving a message
  pingport = int(callargs[0])
  recvmess(getmyip(),pingport,got_message)  
  
  probe_neighbors(pingport)

  # register a function to show a status webpage
  pageport = int(callargs[1])
  waitforconn(getmyip(),pageport,show_status)  
\end{verbatim}
}
%\end{figure}


\balancecolumns
\end{document}



\appendix

The following Seattle program measures network latency to a list of IP
addresses and displays a webpage showing the latency to each
node. This program has 30 lines of code, with 11 lines of
comments. Without the webpage functionality the program is only 21
lines of code. This is a complete Seattle program --- no code is
omitted or abbreviated.

%\caption{This Seattle program pings a list of nodes and gathers latency 
%information.   The user can browse the computer to see a HTML document
%that displays the latency information in a table. }

%\begin{figure}
{
%  # I'm going to write a HTML header first...
%  # and we're done, so let's close this connection...
%  # now the footer...
\scriptsize
\begin{verbatim}
# send a probe message to each neighbor
def probe_neighbors(port):
  for neighborip in mycontext["neighborlist"]:
    mycontext['sendtime'][neighborip] = getruntime()
    sendmess(neighborip, port, 'ping',getmyip(),port)
  # probe again in 10 seconds
  settimer(10,probe_neighbors,(port,))

# handle an incoming message
def got_message(srcip,srcport,mess,ch):
  if mess == 'ping':
    sendmess(srcip,srcport,'pong')
  elif mess == 'pong':
    # elapsed time is now - time when I sent the ping
    mycontext['latency'][srcip] = getruntime() - \
                         mycontext['sendtime'][srcip]

# display a web page with the latency information
def show_status(srcip,srcport,connobj, ch, mainch): 
  connobj.send("<html><head><title>Latency Information</title>"+ \
           "</head><body><h1>Latency information from "+ \
           getmyip()+' </h1><table border="1">')
  # list a row for each node we are talking to 
  for neighborip in mycontext['neighborlist']:
    if neighborip in mycontext['latency']:
      connobj.send("<tr><td>"+neighborip+"</td><td>"+ \
           str(mycontext['latency'][neighborip])+"<td></tr>")
    else:
      connobj.send("<tr><td>"+neighborip+"</td><td>Unknown<td></tr>")
  connobj.send("</table></html>")
  connobj.close()

if callfunc == 'initialize':
  # this holds the response information (i.e. when nodes responded)
  mycontext['latency'] = {}
  # this remembers when we sent a probe
  mycontext['sendtime'] = {}
  # get the nodes to probe
  mycontext['neighborlist'] = []
  for line in file("neighboriplist.txt"):
    mycontext['neighborlist'].append(line.strip())
  # call gotmessage whenever receiving a message
  pingport = int(callargs[0])
  recvmess(getmyip(),pingport,got_message)  
  
  probe_neighbors(pingport)

  # register a function to show a status webpage
  pageport = int(callargs[1])
  waitforconn(getmyip(),pageport,show_status)  
\end{verbatim}
}
%\end{figure}


\balancecolumns
\end{document}



\appendix

The following Seattle program measures network latency to a list of IP
addresses and displays a webpage showing the latency to each
node. This program has 30 lines of code, with 11 lines of
comments. Without the webpage functionality the program is only 21
lines of code. This is a complete Seattle program --- no code is
omitted or abbreviated.

%\caption{This Seattle program pings a list of nodes and gathers latency 
%information.   The user can browse the computer to see a HTML document
%that displays the latency information in a table. }

%\begin{figure}
{
%  # I'm going to write a HTML header first...
%  # and we're done, so let's close this connection...
%  # now the footer...
\scriptsize
\begin{verbatim}
# send a probe message to each neighbor
def probe_neighbors(port):
  for neighborip in mycontext["neighborlist"]:
    mycontext['sendtime'][neighborip] = getruntime()
    sendmess(neighborip, port, 'ping',getmyip(),port)
  # probe again in 10 seconds
  settimer(10,probe_neighbors,(port,))

# handle an incoming message
def got_message(srcip,srcport,mess,ch):
  if mess == 'ping':
    sendmess(srcip,srcport,'pong')
  elif mess == 'pong':
    # elapsed time is now - time when I sent the ping
    mycontext['latency'][srcip] = getruntime() - \
                         mycontext['sendtime'][srcip]

# display a web page with the latency information
def show_status(srcip,srcport,connobj, ch, mainch): 
  connobj.send("<html><head><title>Latency Information</title>"+ \
           "</head><body><h1>Latency information from "+ \
           getmyip()+' </h1><table border="1">')
  # list a row for each node we are talking to 
  for neighborip in mycontext['neighborlist']:
    if neighborip in mycontext['latency']:
      connobj.send("<tr><td>"+neighborip+"</td><td>"+ \
           str(mycontext['latency'][neighborip])+"<td></tr>")
    else:
      connobj.send("<tr><td>"+neighborip+"</td><td>Unknown<td></tr>")
  connobj.send("</table></html>")
  connobj.close()

if callfunc == 'initialize':
  # this holds the response information (i.e. when nodes responded)
  mycontext['latency'] = {}
  # this remembers when we sent a probe
  mycontext['sendtime'] = {}
  # get the nodes to probe
  mycontext['neighborlist'] = []
  for line in file("neighboriplist.txt"):
    mycontext['neighborlist'].append(line.strip())
  # call gotmessage whenever receiving a message
  pingport = int(callargs[0])
  recvmess(getmyip(),pingport,got_message)  
  
  probe_neighbors(pingport)

  # register a function to show a status webpage
  pageport = int(callargs[1])
  waitforconn(getmyip(),pageport,show_status)  
\end{verbatim}
}
%\end{figure}


\balancecolumns
\end{document}
