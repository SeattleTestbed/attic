\section{Related Work}
\label{sec-relatedwork}

%As the
%future of cloud computing is far from clear, Seattle benefits courses
%varying in their coverage of parallel processing and distributed
%computing~\cite{SIGCSE98}.

%Seattle's Python-based interface is simple to learn, and easy to
%master. Yet, it retains much power in the hands of the programmer.
%Seattle can provide a fun experience for students without them having
%to master advanced networking API features nor organize the
%distribution of their code to other platforms or having to organize a
%testbed of machines.

There are many different cloud computing platforms in use today.
Amazon runs a cloud of RedHat servers to provide computing
resources~\cite{AmazonEC2},
which %and a storage back end~\cite{AmazonS3}.
are similar in purpose to Seattle but provide a virtual machine
instead of a programming language instance. This leads to better
performance, but is not flexible to support donated resources and is
not free. Their storage back end is functionally similar to the global
data store proposed as an assignment on Seattle's educator portal.

Microsoft proposes a software plus services~\cite{MicrosoftMemo}
architecture where the cloud is used as an auxiliary to enhance the
capabilities of local software. While Microsoft has announced the
pending availability of a developer toolkit, to the best of our
knowledge it is not available at this time. % for further comparison.

Google provides a cloud computing-like infrastructure with
AppEngine~\cite{GoogleAppEngine}, which executes programs written in a
constrained version of the Python language and supports high level
abstractions (like a global database). While useful for building
locality-oblivious web applications, its transparent handling of
scalability and locality makes it unsuitable for teaching these
fundamental distributed systems topics.

IBM has announced plans for a cloud computing product called ``Blue
Cloud''~\cite{IBMBlueCloud}, which supports OS images using Xen and
PowerVM Linux. Blue Cloud also supports Hadoop~\cite{Hadoop} for
MapReduce-type query processing and is intended to support high
performance computing. %so that organizations can migrate supercomputer
%tasks to Blue Cloud.
Hadoop has also been used to teach cluster computing for large data
processing~\cite{SIGCSE08}. The MapReduce~\cite{MapReduce} paradigm
used by Hadoop simplifies distributed data processing. However, this
simplifying abstraction also limits the scope of systems concepts that
may be taught with Hadoop. We believe that a more general platform
should be used to teach the system concepts that power Hadoop's
implementation. These concepts may then be applied more broadly to
other distributed computing abstractions, and cloud computing more
generally.

In addition to cloud computing, there are a variety of grid computing
and volunteer computing platforms. Globus~\cite{Globus} is a popular
Grid toolkit, which has been used to build a variety of
service-oriented applications. BOINC~\cite{BOINC} is a volunteer
computing platform supporting the SETI@Home and Folding@Home projects.
BOINC leverages donated CPU cycles for computation, particularly spare
resources on home machines. %To BOINC, getting 1\% of the
%time on 1000 computers is similar to getting 100\% on 10 computers
%(for similar computers). With Seattle, it is much more useful to get
%1\% on 1000 computers because the research and educational programs
%are trying to understand how to write scalable programs that run on
%many computers at the same time.
Globus and BOINC both target distributed computation and strive to
hide locality and similar information from the programmer. Seattle is
lighter-weight software that exposes locality and is therefore suited
for students in distributed systems courses. Seattle also comes with a
widely accessible ready-to-use platform of thousands of machines. We
are not aware of any Globus- or BOINC-powered testbeds available for
educational use.

%The problems and
%programs become interesting when there are thousands of computers.

%% Also most volunteer computing platforms like BOINC use CPU cycles only
%% when the system would be idle so as to avoid interfering with the
%% user.  BOINC typically uses almost all of the CPU when it's idle.
%% With Seattle, it is more desirable to constantly have a small
%% percentage of the resources (to allow a more constant set of resources
%% to be used over a longer time period).

Limited compute resources have been a key constraint in teaching
distributed systems~\cite{SIGCSE94}. Seattle is architected to do this
safely and efficiently. % across a variety of architectures.
Recent efforts engaged undergraduates in distributed computing with
simple-to-use platforms designed for cluster
computing. %~\cite{SIGCSE07}.
The DCEZ platform offers a simple interface that students can use
without any prior knowledge of distributed
computing~\cite{SIGCSE07}. %With minimal assistance, students were
%shown to install and use a small cluster with DCEZ within 5 to 10
%minutes.
We have likewise endeavored to make Seattle simple to use, but
target teaching of distributed systems issues that arise at Internet
scales. Lastly, because Seattle runs on a variety of embedded
platforms with limited resources, such as cellular phones and PDAs,
our platform complements prior work on platforms for teaching
ubiquitous computing~\cite{SIGCSE03}.

